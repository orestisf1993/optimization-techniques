\chapter{Υλοποίηση στο Matlab} \label{ch:matlab}

Για λόγους γρηγορότερης εκτέλεσης τα αρχεία είναι vectorized,
δηλαδή μπορούν αν δέχονται σαν ορίσματα διανύσματα αντί για βαθμωτά μεγέθη.

\section{gaussianSum.m}

\lstinputlisting[language=Matlab, caption=gaussianSum.m]{../gaussianSum.m}

Ο κώδικας για την υλοποίηση του αθροίσματος των γκαουσιανών συναρτήσεων βρίσκεται στο αρχείο gaussianSum.m. Επιστρέφεται η τιμή 
\begin{equation}
	z = \sum_{i = 1}^{nGaussian} a_i \cdot e^{-[\frac{(u_1 - c_{1_i})^2}{2\sigma_{1_i}^2} + \frac{(u_2 - c_{2_i})^2}{2\sigma_{2_i}^2}]}
\end{equation}

Το όρισμα $x$ περιέχει το διάνυσμα με τα μεγέθη που χρησιμοποιούνται για τον προσδιορισμό των συναρτήσεων 
και είναι αυτό που θα προσπαθήσουμε να βελτιστοποιήσουμε μέσω της ga().

Αν ορίζουμε ως $nGaussian$ τον αριθμό των γκαουσιανών συναρτήσεων που χρησιμοποιούμε, 
τότε τα πρώτα $nGaussian$ στοιχεία του $x$ είναι οι συντελεστές $a_i$, 
τα επόμενα $2 \cdot nGaussian$ στοιχεία είναι οι συντελεστές $\sigma_{1_i}$ και $\sigma_{2_i}$ 
και τα τελευταία $2 \cdot nGaussian$ στοιχεία είναι οι συντελεστές $c_{1_i}$ και $c_{2_i}$.

Λόγω του vectorization της συνάρτησης ως προς $x$ αλλά και ως προς $u_1$, $u_2$ χρησιμοποιείται η συνάρτηση bsxfun() του Matlab ώστε να γίνονται πράξεις στοιχείο προς στοιχείο για ασύμμετρους πίνακες.

\section{error\_sum.m}

\lstinputlisting[language=Matlab, caption=error\_sum.m]{../error_sum.m}

Σχετικά απλή συνάρτηση που αθροίζει τα τετράγωνα των διαφορών $f(u_1, u_2)$ και $gaussianSum()$.

\section{run\_all.m}

\lstinputlisting[language=Matlab, caption=run\_all.m]{../run_all.m}

Στη φάση του initialization ορίζουμε τα διανύσματα $u_1$, $u_2$ και θέτουμε τα όρια του διανύσματος $x$. 
Συγκεκριμένα, τα $\sigma_{i}$ επιλέγονται θετικά και μικρότερα του $2\sqrt{2}$. 
Η επιλογή αυτή έγινε καθώς τα $\sigma$ σχετίζεται ουσιαστικά με το πλάτος της γκαουσιανής "καμπάνας" ως προς $u_1$ και $u_2$ και δεν θα είχε νόημα αν αυτή ήταν μεγαλύτερη από το πεδίο ορισμού $[-2, 2]$. 
Θέλουμε τουλάχιστον η μισή "καμπάνα" να είναι μέσα στο πεδίο ορισμού έχουμε $\sigma > 0$ και $\frac{\sigma^2}{2} \leq 2 - (-2) \implies 0 < \sigma \leq 2\sqrt{2}$. 
Επίσης, επιλέγουμε τα $c_i$ που καθορίζουν την θέση των κορυφών τέτοια ώστε να είναι μέσα στο πεδίο ορισμού, δηλαδή $-2 \leq c \leq 2$. Δοκιμάστηκαν και μεγαλύτερα όρια ($-6 \leq c \leq 6$) αλλά είχαν χειρότερα αποτελέσματα.
Τα $a_i$ δεν έχουν όρια.

Στη συνέχεια, καλούμε την ga() δοκιμάζοντας αριθμό γκαουσιανών από 1 έως 15.
Στην κάθε κλίση της ga() ελαχιστοποιούμε ως προς ένα διάνυσμα $x$ $5 \cdot nGaussian$ στοιχείων. 
Τα $u_1$ και $u_2$ μένουν σταθερά σύμφωνα με τον αρχικό υπολογισμό τους.

Τελικά, επιλέγουμε έναν μια από τις τιμές $nGaussian$ και προσπαθούμε να βελτιώσουμε την προσέγγιση περαιτέρω ορίζοντας επαναληπτικά ως αρχικό πληθυσμό της ga() το προηγούμενο αποτέλεσμα $x$.