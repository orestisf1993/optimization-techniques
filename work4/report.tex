% Specify the type of document
\documentclass[12pt,titlepage,letterpaper,onecolumn]{report}

% User packages

% Only for xelatex and lualatex. It provides an automatic and unified interface to feature-rich AAT and OpenType fonts.
% https://ctan.org/pkg/fontspec
\usepackage{fontspec}
% https://www.ctan.org/pkg/inputenc
\usepackage[utf8]{inputenc}

% Select alternative section titles
% https://www.ctan.org/pkg/titlesec
\usepackage{titlesec}

% For graphics. Used for \usepackage{}
% http://www.kwasan.kyoto-u.ac.jp/solarb6/usinggraphicx.pdf
% https://ctan.org/pkg/graphicx
\usepackage{graphicx}

% provides LaTeX the ability to create hyperlinks within the document.
% https://www.ctan.org/pkg/hyperref
% http://en.wikibooks.org/wiki/LaTeX/Hyperlinks#Hyperref
\usepackage{hyperref}
% for going to the top of an image when a figure reference is clicked: http://stackoverflow.com/a/21251925/3430986%
\usepackage{caption}

% http://tex.stackexchange.com/questions/118713/is-microtype-fully-supported-now-by-xelatex-if-not-how-can-i-keep-myself-up-to
% http://tug.org/pipermail/xetex/2013-April/024263.html
\usepackage{microtype}
\usepackage{amsmath}
\usepackage{amsfonts}
\usepackage{fancyref}
\usepackage{multicol}
\newenvironment{Figure}
  {\par\medskip\noindent\minipage{\linewidth}}
  {\endminipage\par\medskip}
\usepackage[margin=1in]{geometry}

% Used for pgf plots. Taken from https://github.com/matlab2tikz/matlab2tikz
% also see http://www.howtotex.com/packages/beautiful-matlab-figures-in-latex/
\usepackage{pgfplots}
\pgfplotsset{compat=newest}
\pgfplotsset{plot coordinates/math parser=false}
\newlength\figureheight
\newlength\figurewidth

%\usetikzlibrary{external}
%\tikzexternalize[prefix=tikz/]

% Quickly check the document. To produce pages, comment it out
\usepackage{syntonly}
\syntaxonly

\titleformat{\chapter}[hang]{\bf\huge}{\thechapter}{2pc}{}
\setmainfont{DejaVu Serif}
% plain, headings or empty. Headins prints the current chapter heading + page number
\pagestyle{headings}

% http://tex.stackexchange.com/a/27099
\makeatletter
\newcommand*{\centerfloat}{%
  \parindent \z@
  \leftskip \z@ \@plus 1fil \@minus \textwidth
  \rightskip\leftskip
  \parfillskip \z@skip}
\makeatother

% Change the title of contents list and figure list
\renewcommand{\contentsname}{Περιεχόμενα}
\renewcommand{\listfigurename}{Λίστα Σχημάτων}
% Change the name of each figure
\renewcommand{\figurename}{Σχήμα}

% defines a variable
%\def \variableName {Something that's better to use as a variable}

% \\ or \newline == newline, no paragraph
% \\* == no page break

\title{4η Εργαστηριακή Άσκηση}
\author{Ορέστης Φλώρος-Μαλιβίτσης\\
  Τομέας Ηλεκτρονικής,\\
  Τμήμα Ηλ. Μηχανικών / Μηχανικών ΗΥ,\\
  Αριστοτέλειο Πανεπιστήμιο Θεσσαλονίκης}
\date{22/05/2015}

\begin{document}
\maketitle
\tableofcontents
\listoffigures
\newpage


% Δομή του Project
\chapter*{Δομή του Project} \label{project-structure}

\begin{description}
	\item[Floros\_orestis\_Work4\_7796.pdf] Αυτή η αναφορά.
	\item[work4.pdf] Η εκφώνηση της εργαστηριακής άσκησης.
	\item[plots] Φάκελος που βρίσκονται οι διάφορες γραφικές παραστάσεις σε μορφή png. Επίσης, περιέχει τους κώδικες Matlab για την δημιουργία αυτών των γραφικών παραστάσεων.
	\item[functions] Φάκελος με την συνάρτηση και την παράγωγό της.
\end{description}

% Εισαγωγή
\chapter{Εισαγωγή} \label{intro}

Σε αυτή την άσκηση θεωρούμε την τετραγωνική συνάρτηση
\begin{equation*}
	f:\mathbb{R}^{2} \mapsto \mathbb{R}, \qquad f(x) = \frac{1}{2} \cdot x_1 ^ 2 + \frac{1}{2} \cdot x_2^2
\end{equation*}
Υπολογίζουμε την κλίση $\nabla f(x, y) = \{x, y\}$. \newline
Στα σχήματα \ref{fig:000f_mesh_plot} με \ref{fig:003f_contour_plot_fill} παρουσιάζονται διάφορες γραφικές παραστάσεις της f ώστε να έχουμε μία γεωμετρική ιδέα για την μορφή της. Στα σχήματα \ref{fig:006gradf_quiver_f_contour_plot} και \ref{fig:005gradf_mesh_plot} παρουσιάζονται γραφήματα για την $\nabla f$.

% These figures will be showed in a multicol style. 2 figures per row.
\begin{multicols}{2}

\begin{Figure}
%	\centerfloat
	\centering
	\setlength\figureheight{6cm}
	\setlength\figurewidth{6cm} 
	\input{plots/functions/000f_mesh_plot}
	\captionsetup{type=figure}
	\caption{$f(x,y)$ mesh plot}
	\label {fig:000f_mesh_plot}
\end{Figure}


\begin{Figure}
	\centering
	\setlength\figureheight{6cm}
	\setlength\figurewidth{6cm}
	\input{plots/functions/001f_surf_plot}
	\captionsetup{type=figure}
	\caption{$f(x,y)$ surface plot}
	\label {fig:001f_surf_plot}
\end{Figure}


\begin{Figure}
	\centering
	\setlength\figureheight{6cm}
	\setlength\figurewidth{6cm}
	\input{plots/functions/002f_contour_plot_showtext}
	\captionsetup{type=figure}
	\caption{$f(x,y)$ contour plot}
	\label{fig:002f_contour_plot_showtext}
\end{Figure}

\begin{Figure}
	\centering
	\setlength\figureheight{6cm}
	\setlength\figurewidth{6cm}
	\input{plots/functions/003f_contour_plot_fill}
	\captionsetup{type=figure}
	\caption{$f(x,y)$ contour plot}
	\label{fig:003f_contour_plot_fill}
\end{Figure}


\begin{Figure}
	\centering
	\setlength\figureheight{6cm}
	\setlength\figurewidth{6cm}
	\input{plots/functions/006gradf_quiver_f_contour_plot}
	\captionsetup{type=figure}
	\caption{quiver plot of $\parallel \nabla f(x,y) \parallel$ with contour plot of $f(x,y)$}
	\label{fig:006gradf_quiver_f_contour_plot}
\end{Figure}


\begin{Figure}
	\centering
	\setlength\figureheight{6cm}
	\setlength\figurewidth{6cm}
	\input{plots/functions/005gradf_mesh_plot}
	\captionsetup{type=figure}
	\caption{$\parallel \nabla f(x,y) \parallel$ mesh plot}
	\label{fig:005gradf_mesh_plot}
\end{Figure}

% continue using 1 col per page.
\end{multicols}


\chapter{Ερώτημα (α)}


\end{document}
