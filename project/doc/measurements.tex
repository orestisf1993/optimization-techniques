\chapter{Μετρήσεις και αποτελέσματα} \label{ch:meashurements}

Λόγω της τυχαιότητας των γενετικών αλγορίθμων τα αποτελέσματα που βρίσκουμε δεν είναι σταθερά και ο αριθμός των γκαουσιανών για τον οποίο έχουμε μικρότερο error αλλάζει κάθε φορά που εκτελούμε τον κώδικα.
\newline

\begin{tabular}{ c | c }
   nGaussian & error \\
   \hline
	1 & 128.9957 \\ 
	2 & 61.6691 \\ 
	3 & 50.8957 \\ 
	4 & 35.9179 \\ 
	5 & 42.1853 \\ 
	6 & 48.8752 \\ 
	7 & 23.265 \\ 
	8 & 11.9985 \\ 
	9 & 20.0387 \\ 
	10 & 14.7827 \\ 
	11 & 61.483 \\ 
	12 & 20.3933 \\ 
	13 & 13.0228 \\ 
	14 & 6.3402 \\ 
	15 & 6.3388 \\ 
\end{tabular}
\captionof{table}{Αποτελέσματα για το error σε σχέση με τον αριθμό των γκαουσιανών}

\newpage
Παρουσιάζονται οι γραφικές παραστάσεις των προσεγγίσεων.
\begin{multicols}{2}

\begin{Figure}
	\includegraphics[width=5.5cm]{adapted_1}
	\label{fig:adapted-1}
\end{Figure}
\begin{Figure}
	\includegraphics[width=5.5cm]{adapted_2}
	\label{fig:adapted-2}
\end{Figure}
\begin{Figure}
	\includegraphics[width=5.5cm]{adapted_3}
	\label{fig:adapted-3}
\end{Figure}
\begin{Figure}
	\includegraphics[width=5.5cm]{adapted_4}
	\label{fig:adapted-4}
\end{Figure}
\begin{Figure}
	\includegraphics[width=5.5cm]{adapted_5}
	\label{fig:adapted-5}
\end{Figure}
\begin{Figure}
	\includegraphics[width=5.5cm]{adapted_6}
	\label{fig:adapted-6}
\end{Figure}
\begin{Figure}
	\includegraphics[width=5.5cm]{adapted_7}
	\label{fig:adapted-7}
\end{Figure}
\begin{Figure}
	\includegraphics[width=5.5cm]{adapted_8}
	\label{fig:adapted-8}
\end{Figure}
\begin{Figure}
	\includegraphics[width=5.5cm]{adapted_9}
	\label{fig:adapted-9}
\end{Figure}
\begin{Figure}
	\includegraphics[width=5.5cm]{adapted_10}
	\label{fig:adapted-10}
\end{Figure}
\begin{Figure}
	\includegraphics[width=5.5cm]{adapted_11}
	\label{fig:adapted-11}
\end{Figure}
\begin{Figure}
	\includegraphics[width=5.5cm]{adapted_12}
	\label{fig:adapted-12}
\end{Figure}
\begin{Figure}
	\includegraphics[width=5.5cm]{adapted_13}
	\label{fig:adapted-13}
\end{Figure}
\begin{Figure}
	\includegraphics[width=5.5cm]{adapted_14}
	\label{fig:adapted-14}
\end{Figure}

\end{multicols}

\begin{figure}[bh]
	\centerfloat
	\includegraphics[width=8cm]{adapted_15}
	\label{fig:adapted-15}
\end{figure}

\newpage
Παρουσιάζονται οι γραφικές παραστάσεις των error.
\begin{multicols}{2}

\begin{Figure}
	\includegraphics[width=5.5cm]{adapted_error_1}
	\label{fig:adapted-error-1}
\end{Figure}
\begin{Figure}
	\includegraphics[width=5.5cm]{adapted_error_2}
	\label{fig:adapted-error-2}
\end{Figure}
\begin{Figure}
	\includegraphics[width=5.5cm]{adapted_error_3}
	\label{fig:adapted-error-3}
\end{Figure}
\begin{Figure}
	\includegraphics[width=5.5cm]{adapted_error_4}
	\label{fig:adapted-error-4}
\end{Figure}
\begin{Figure}
	\includegraphics[width=5.5cm]{adapted_error_5}
	\label{fig:adapted-error-5}
\end{Figure}
\begin{Figure}
	\includegraphics[width=5.5cm]{adapted_error_6}
	\label{fig:adapted-error-6}
\end{Figure}
\begin{Figure}
	\includegraphics[width=5.5cm]{adapted_error_7}
	\label{fig:adapted-error-7}
\end{Figure}
\begin{Figure}
	\includegraphics[width=5.5cm]{adapted_error_8}
	\label{fig:adapted-error-8}
\end{Figure}
\begin{Figure}
	\includegraphics[width=5.5cm]{adapted_error_9}
	\label{fig:adapted-error-9}
\end{Figure}
\begin{Figure}
	\includegraphics[width=5.5cm]{adapted_error_10}
	\label{fig:adapted-error-10}
\end{Figure}
\begin{Figure}
	\includegraphics[width=5.5cm]{adapted_error_11}
	\label{fig:adapted-error-11}
\end{Figure}
\begin{Figure}
	\includegraphics[width=5.5cm]{adapted_error_12}
	\label{fig:adapted-error-12}
\end{Figure}
\begin{Figure}
	\includegraphics[width=5.5cm]{adapted_error_13}
	\label{fig:adapted-error-13}
\end{Figure}
\begin{Figure}
	\includegraphics[width=5.5cm]{adapted_error_14}
	\label{fig:adapted-error-14}
\end{Figure}

\end{multicols}

\begin{figure}[bh]
	\centerfloat
	\includegraphics[width=8cm]{adapted_error_15}
	\label{fig:adapted-error-15}
\end{figure}

\hfill \break
Για να μην αυξήσουμε πολύ την πολυπλοκότητα της έκφρασης της $f$ διαλέγουμε έναν σχετικά μικρό αριθμό γκαουσιανών για να μελετήσουμε παραπάνω. 
Για $nGaussian = 8$ επαναλαμβάνουμε την εύρεση ελαχίστου μέχρι που ρίχνουμε το error από $11.9985$ σε $10.6045$. 
Τα νέα αποτελέσματα φαίνονται στα γραφήματα \ref{fig:adapted-opt-8} και \ref{fig:adapted-opt-error-8}. 
Ωστόσο, όπως φαίνεται στα γραφήματα \ref{fig:adapted-actual-8} και \ref{fig:adapted-error-actual-8} παρατηρείται μια κορύφωση του σφάλματος, 
που δεν φαίνεται στα προηγούμενα γραφήματα, 
όταν χρησιμοποιείται διαφορετικό, πυκνότερο σύνολο σημείων για τον σχεδιασμό του γραφήματος.

\begin{multicols}{2}

\begin{Figure}
	\centerfloat
	\includegraphics[width=5.5cm]{adapted_opt_8}
	\captionof{figure}{Βελτιωμένη προσέγγιση 8 γκαουσιανών}
	\label{fig:adapted-opt-8}
\end{Figure}

\begin{Figure}
	\centerfloat
	\includegraphics[width=5.5cm]{adapted_opt_error_8}
	\captionof{figure}{error βελτιωμένης προσέγγιση 8 γκαουσιανών}
	\label{fig:adapted-opt-error-8}
\end{Figure}

\begin{Figure}
	\centerfloat
	\includegraphics[width=5.5cm]{adapted_actual_8}
	\captionof{figure}{Βελτιωμένη προσέγγιση 8 γκαουσιανών με διαφορετικά σημεία}
	\label{fig:adapted-actual-8}
\end{Figure}

\begin{Figure}
	\centerfloat
	\includegraphics[width=5.5cm]{adapted_actual_error_8}
	\captionof{figure}{error βελτιωμένης προσέγγιση 8 γκαουσιανών με διαφορετικά σημεία}
	\label{fig:adapted-error-actual-8}
\end{Figure}

\end{multicols}
