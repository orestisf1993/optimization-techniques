%&preamble
% extra for preamble:
\usepackage{enumerate}% http://ctan.org/pkg/enumerate
%% end of static part. Save at preamble.tex and use command: xelatex -ini -shell-escape -job-name="preamble" "&xelatex preamble.tex\dump"
%% dynamic part

% Only for xelatex and lualatex. It provides an automatic and unified interface to feature-rich AAT and OpenType fonts.
% https://ctan.org/pkg/fontspec
\usepackage{fontspec}
\setmainfont{DejaVu Serif}
% Change the title of contents list and figure list
\renewcommand{\contentsname}{Περιεχόμενα}
\renewcommand{\listfigurename}{Λίστα Σχημάτων}
% Change the name of each figure
\renewcommand{\figurename}{Σχήμα}

% defines a variable
%\def \variableName {Something that's better to use as a variable}

% \\ or \newline == newline, no paragraph
% \\* == no page break

\title{4η Εργαστηριακή Άσκηση}
\author{Ορέστης Φλώρος-Μαλιβίτσης\\
  Τομέας Ηλεκτρονικής,\\
  Τμήμα Ηλ. Μηχανικών / Μηχανικών ΗΥ,\\
  Αριστοτέλειο Πανεπιστήμιο Θεσσαλονίκης}
\date{22/05/2015}

\includeonly{part_a}

% Quickly check the document. To produce pages, comment it out
%\usepackage{syntonly}
%\syntaxonly

\begin{document}
\maketitle
\tableofcontents
\listoffigures
\newpage


% Δομή του Project
\chapter*{Δομή του Project} \label{project-structure}

\begin{description}
	\item[Floros\_orestis\_Work4\_7796.pdf] Αυτή η αναφορά.
	\item[work4.pdf] Η εκφώνηση της εργαστηριακής άσκησης.
	\item[plots] Φάκελος που βρίσκονται οι διάφορες γραφικές παραστάσεις σε μορφή png. Επίσης, περιέχει τους κώδικες Matlab για την δημιουργία αυτών των γραφικών παραστάσεων.
	\item[functions] Φάκελος με την συνάρτηση και την παράγωγό της.
\end{description}

% Εισαγωγή
\chapter{Εισαγωγή} \label{intro}

Δίνονται οι συναρτήσεις για ελαχιστοποίηση:
\begin{equation}
	\label{eq:f-definition}
	f:\mathbb{R}^{2} \mapsto \mathbb{R}, \qquad f(x_1, x_2) = x_1 \cdot x_2 + 2 \cdot (x_1 - x_2)^2
\end{equation}
με περιορισμούς
\begin{align}
	3 \leq x_1& \leq 30\\
	-25 \leq x_2& \leq -50
\end{align}
και
\begin{equation}
	\label{eq:g-definition}
	g:\mathbb{R}^{2} \mapsto \mathbb{R}, \qquad g(x_1, x_2) = (x_1 - x_2)^2
\end{equation}
με περιορισμούς
\begin{align}
	x_1& \leq -1\\
	x_2& \leq -1
\end{align}
Υπολογίζονται οι κλίσεις τους:

\begin{equation}
\nabla f(x_1, x_2) = 
\begin{bmatrix}
	4 \cdot x_1 - 3 \cdot x_2\\
	-3 \cdot x_1 + 4 \cdot x_2
\end{bmatrix}
\end{equation}
και
\begin{equation}
\nabla g(x_1, x_2) = 
\begin{bmatrix}
	2 \cdot (x_1 - x_2)\\
	-2 \cdot (x_1 - x_2)
\end{bmatrix}
\end{equation}
Οι Εσσιανοί πίνακες υπολογίζονται ως:
\begin{equation}
\nabla^2 f(x_1, x_2) = 
\begin{bmatrix}
	4 & -3\\
	-3 & 4
\end{bmatrix}
\end{equation} με ιδιοτιμές $\lambda_1 = 7$ και $\lambda_2 = 1$.
\begin{equation}
\nabla^2 f(x_1, x_2) = 
\begin{bmatrix}
	2 & -2\\
	-2 & 2
\end{bmatrix}
\end{equation} με ιδιοτιμές $\lambda_1 = 4$ και $\lambda_2 = 0$.\\
Παρατηρούμε ότι λόγω των Εσσιανών τους πινάκων οι $f$ και $g$ είναι κυρτές συναρτήσεις και τα αντίστοιχα προβλήματα είναι κυρτά και υπερ-αποτελούμενα.\\
Παρουσιάζονται σχετικές γραφικές παραστάσεις.

\setlength{\LTleft}{-20cm plus -1fill}
\setlength{\LTright}{\LTleft}

\section{Συνάρτηση f}
\begin{longtable}{cc}
	\centering
	\includegraphics[width=100mm]{plots/functions/f_mesh_plot.eps}&
	\includegraphics[width=100mm]{plots/functions/f_mesh_plot_limited.eps}\\
	\includegraphics[width=100mm]{plots/functions/f_contourf_plot.eps} &
	\includegraphics[width=100mm]{plots/functions/f_contourf_plot_limited.eps}\\
	\includegraphics[width=100mm]{plots/functions/gradf_mesh_plot.eps} &
	\includegraphics[width=100mm]{plots/functions/gradf_mesh_plot_limited.eps}\\
\end{longtable}

\newpage
\section{Συνάρτηση g}
\setlength{\LTleft}{-20cm plus -1fill}
\setlength{\LTright}{\LTleft}
\begin{longtable}{cc}
	\centerfloat
	\includegraphics[width=100mm]{plots/functions/g_mesh_plot.eps} &
	\includegraphics[width=100mm]{plots/functions/g_mesh_plot_limited.eps}\\
	\includegraphics[width=100mm]{plots/functions/g_contourf_plot.eps} &
	\includegraphics[width=100mm]{plots/functions/g_contourf_plot_limited.eps}\\
	\includegraphics[width=100mm]{plots/functions/gradg_mesh_plot.eps} &
	\includegraphics[width=100mm]{plots/functions/gradg_mesh_plot_limited.eps}\\
\end{longtable}
\chapter{Ερώτημα (α)}

Με την χρήση της μεθόδου της μεγίστης καθόδου με σταθερό βήμα $\gamma_k$ (υλοποίηση της προηγούμενης εργασίας) 
αναζητούμε το ελάχιστο της συνάρτησης με αρχικό σημείο το (-5,5).
Στο σχήμα \ref{fig:stepeest_plots} βλέπουμε τα αποτελέσματα για $\gamma = 0.1$, $\gamma = 1$, $\gamma = 2$, $\gamma = 10$.

\begin{figure}[hbtp]
	\centering
	\setlength\figureheight{\textwidth}
	\setlength\figurewidth{\textwidth}
%	\input{plots/algorithms/stepeest_plots.tikz}
	\includetikz{plots/algorithms/stepeest_plots}
	\caption{Σύγκλιση της μεγίστης καθόδου για σταθερά γ}
	\label{fig:stepeest_plots}
\end{figure}

\noindent \begin{minipage}{\linewidth}
\noindent Παρατηρούμε ότι:
\begin{enumerate}[i)]
	\item Για $\gamma = 0.1$ ο αλγόριθμος συγκλίνει επιτυχώς στο ελάχιστο σε 64 βήματα.
	\item Για $\gamma = 1$ ο αλγόριθμος συγκλίνει επιτυχώς στο ελάχιστο σε 2 βήματα. Δηλαδή, από το σημείο (-5, 5) καταλήγει αμέσως στη λύση (0,0).
	\item Για $\gamma = 2$ ο αλγόριθμος δεν συγκλίνει. Παρατηρούμε μια "ταλάντωση" μεταξύ του αρχικού σημείου (-5, 5) και του (5, -5). Δηλαδή, ο αλγόριθμος εναλλάσσεται μεταξύ των δύο αυτών σημείων σε κάθε βήμα και δεν καταλήγει ποτέ στο ελάχιστο.
	\item Για $\gamma = 10$ ο αλγόριθμος απομακρύνεται συνεχόμενα από το αρχικό του σημείο. Η ακολουθία $x_k$ παίρνει τιμές $(-5, 5) \rightarrow (45, -45) \rightarrow (-405, 405) \rightarrow (3645, -3645) \rightarrow ... \rightarrow (\infty, -\infty)$ (όπου $\infty$ η μέγιστες τιμές που μπορούν να χρησιμοποιηθούν στο Matlab).
\end{enumerate}
\end{minipage}

\hfill \break
\hfill \break
\noindent Μια εξήγηση μπορεί να δοθεί από τον τύπο για την ακολουθία της μεγίστης καθόδου: 
\begin{equation}
	x_{k + 1} = x_k - \gamma_k \cdot \nabla f(x_k) 
\end{equation}
όπου αν αντικαταστήσουμε την τιμή του $\nabla f(x_k) = x_k$ και θεωρήσουμε σταθερό $\gamma_k = \gamma$ έχουμε:
\begin{equation}
	x_{k + 1} = x_k - \gamma \cdot x_k = (1- \gamma) \cdot x_k
\end{equation}
\hfill \break
\noindent Έτσι, αντικαθιστώντας για τις διάφορες τιμές του $\gamma$ έχουμε:
\begin{enumerate}[i)]
	\item $\gamma = 0.1 \implies x_{k + 1} = 0.9 \cdot x_k$
	
	Οπότε, αν έχουμε ένα αρχικό σημείο $x_1$ η ακολουθία $x_k$ έχει ως εξής:
	\begin{itemize}
		\item $x_2 = 0.9 \cdot x_1$
		\item $x_3 = 0.9 \cdot x_2 = 0.9^2 \cdot x_1$
		\item $x_4 = 0.9 \cdot x_3 = 0.9^3 \cdot x_1$
		\item \ldots
		\item $x_k = 0.9^{k-1} \cdot x_1$
	\end{itemize}
	
	Παρατηρούμε ότι $\norm{x_{k+1}} < \norm{x_{k}} \forall k \in \mathbb{N}$ και ότι η ακολουθία συγκλίνει προς το (0,0) για $k \to +\infty$.
	
	Ισχύει
	\begin{equation}
		(x_1, x_2) = \frac{1}{2} \cdot (x_1^2 + x_2^2) = \frac{1}{2} \cdot \norm{(x_1, x_2)}
	\end{equation} άρα	$\norm{x_{k+1}} < \norm{x_{k}} \implies f(x_{k+1}) < f(x_k)$.
	
	Ακόμα, έχουμε 
	\begin{equation}
		\norm{\nabla f(x_k)} = \norm{x_k} = 0.9^{k-1} \cdot \norm{x_1}
	\end{equation}
	και θέλουμε 
	\begin{equation}
		\label{eq:while_termination}
		\norm{\nabla f(x_k)} < \epsilon
	\end{equation}
	
	Λύνουμε την εξίσωση \[ 0.9^{k-1} \cdot \norm{x_1} = \epsilon \] 
	παίρνοντας λογαρίθμους και στις 2 μεριές: 
	\begin{align*}
		 (k - 1) \cdot \ln 0.9 &= ln(\frac{\epsilon}{\norm{x_1}}) \implies \\
		 k &= \frac{\ln \epsilon - \ln  \norm{x_1}}{\ln 0.9} + 1
	\end{align*}
	Τελικά, αντικαθιστώντας για $\epsilon = 0.01$ και $x_1 = (-5, 5)$ παίρνουμε $k = 63.27$
	
	Αφού $\norm{\nabla f(x_{k+1})} < \norm{\nabla f(x_k)}$ η σχέση \ref{eq:while_termination} ικανοποιείται για $k = \left \lceil{63.27}\right \rceil$
	δηλαδή σε 64 βήματα το οποίο επαληθεύει τα αποτελέσματα από την εκτέλεση του κώδικα που βρήκαμε πριν.
	
	\item $\gamma = 1 \implies x_{k+1} = (0,0)$
	
	Οπότε, άσχετα με την επιλογή του $x_1$ θα έχουμε $x_2 = 0$ το οποίο επαληθεύει και τα αποτελέσματα από το Matlab.
	
	Για να διερευνήσουμε παραπάνω τον λόγο που συμβαίνει αυτό, παραθέτουμε το εξής χωρίο από την σελίδα 123 του βιβλίου:
	\begin{quote}
		Η μόνη περίπτωση ο αλγόριθμος της μεγίστης καθόδου να μας οδηγήσει στο ελάχιστο $x^*$ σ' ένα βήμα, είναι όταν το διάνυσμα που συνδέει το $x_k$ με το $x^*$ είναι συγγραμικό με το $\nabla f(x_k)$. Αυτό βέβαια οφείλεται στη χρήση της υπορουτίνας που ελαχιστοποιεί την $f(x_k - \gamma_k \cdot \nabla f(x_k)))$ ως προς το $\gamma_k$ \ldots
	\end{quote}
	
	Επειδή $x_k - x^* = x_k - (0,0) = x_k = \nabla f(x_k)$ το διάνυσμα που συνδέει το $x_k$ με το $x^*$ είναι συγγραμικό με το $\nabla f(x_k)$.\\
	
	Επίσης,
	\begin{gather*}
		f(x_k - \gamma_k \cdot \nabla f(x_k))) =\\
		f(x_k - \gamma_k \cdot x_k)	=\\
		f((1-\gamma_k) \cdot x_k) =\\
		\frac{1}{2} \cdot (1 - \gamma_k)^2 \cdot x_1^2 + \frac{1}{2} \cdot (1 - \gamma_k)^2 \cdot x_2^2 =\\
		(1 - \gamma)^2 \cdot f(x_k)
	\end{gather*}
	η οποία ελαχιστοποιείται πάντα για $\gamma = 1$ άσχετα με το σημείο $x_k$.

	Άρα, για οποιαδήποτε επιλογή σημείου $x_1$ θα ισχύει 
	\[\argmin f(x_k - \gamma_k \cdot \nabla f(x_k))) = 1 = \gamma\]
	και
	\[x^* \parallel \nabla f(x_k)\]
	
 	\item $\gamma = 2 \implies x_{k+1} = -x_{k}$
 	
 	Έτσι, $x_{k+2}=-x_{k+1}=x_{k}$. 
 	Αυτό εξηγεί την συμπεριφορά του αλγορίθμου δηλαδή την "ταλάντωση" μεταξύ του αρχικού σημείου και του συμμετρικού του ως προς (0,0).
 	\item $\gamma = 10  \implies x_{k+1} = -9 \cdot x_k$
 		
	Οπότε, αν έχουμε ένα αρχικό σημείο $x_1$ η ακολουθία $x_k$ έχει ως εξής:
	\begin{itemize}
		\item $x_2 = -9 \cdot x_1$
		\item $x_3 = -9 \cdot x_2 = (-9)^2 \cdot x_1$
		\item $x_4 = -9 \cdot x_3 = (-9)^3 \cdot x_1$
		\item \ldots
		\item $x_k = (-9)^{k-1} \cdot x_1$
	\end{itemize}
	Παρατηρούμε ότι $\norm{x_{k+1}} > \norm{x_{k}} \forall k \in \mathbb{N}$ και ότι η ακολουθία αποκλίνει από το (0,0) για $k \to +\infty$.
	
	Αντικαθιστώντας για $\epsilon = 0.01$ και $x_1 = (-5, 5)$:
	\begin{itemize}
		\item $x_2 = (45, -45)$
		\item $x_3 = (-405, 405)$
		\item $x_4 = (3645, -3645)$
		\item ...
	\end{itemize}
	Έτσι, εξηγείται η συμπεριφορά του αλγορίθμου και για αυτή τη τιμή του $\gamma$.
\end{enumerate}

\hfill \break
\hfill \break
Γενικά, για να συγκλίνει ο αλγόριθμος θα πρέπει να ισχύει:
\begin{gather*}
	\norm{x_{k+1}} < \norm{x_k} \implies\\
	\norm{(1 - \gamma) \cdot x_k} < \norm{x_k} \xLongrightarrow{x_k \neq (0,0)}\\
	\abs{1 - \gamma} < 1 \implies\\
	0 < \gamma < 2
\end{gather*}
\subsection{Ερώτημα (β)}


Χρησιμοποιούμε τα ορίσματα $s_k = 15$, $\gamma_k = 0.1$, $\epsilon = 0.01$ και αρχικό σημείο το $(8,3)$.
Οι σχέσεις \ref{eq:x1-final} και \ref{eq:x2-final} γίνονται:
\begin{align}
	\label{eq:x1-finalb}
	x_{1_{k+1}} =		
		\begin{cases}
			0.9 \cdot x_{1_{k}} - 2, &$ αν $ x_{1_{k}} \geq \frac{10}{7} \\
			-0.5 \cdot x_{1_{k}}, &$ αν $ -\frac{5}{7} < x_{1_{k}} < \frac{10}{7}\\
			0.9 \cdot x_{1_{k}} + 1 \cdot, &$ αν $ x_{1_{k}} \leq -\frac{5}{7}
		\end{cases}
\end{align}
και
\begin{align}
	\label{eq:x2-finalb}
	x_{2_{k+1}} = 	
		\begin{cases}
			0.9 \cdot x_{2_{k}} - 1.2, &$ αν $ x_{2_{k}} \geq \frac{6}{7} \\
			-0.5 \cdot x_{2_{k}}, &$ αν $ -\frac{15}{14} < x_{2_{k}} < \frac{6}{7} \\
			0.9 \cdot x_{2_{k}} + 1.5 \cdot, &$ αν $ x_{2_{k}} \leq -\frac{15}{14}
		\end{cases}
\end{align}

Παρατηρούμε ότι άν το $-14 \cdot x_k \in X$ τότε $x_{k+1} = -0.5 \cdot x_k$. 
Για τους συγκεκριμένους περιορισμούς για οποιοδήποτε $x_k \in X$ ισχύει $-0.5 \cdot x_k \in X$ άρα αν $-14 \cdot x_k \in X$ τότε $-14 \cdot x_{k+1} \in X$. 
Έτσι, αν για κάποιο $k$ έχουμε $x_k \in X$ τότε $\forall k' > k$ θα ισχύει $x_{k'} \in X$.

Τα $s$ και $\gamma$ ικανοποιούν την σχέση \ref{eq:s-gamma-requirements}.
Όταν εκτελούμε τον αλγόριθμο τερματίζεται για $k = 15$ (για αρχικό σημείο $x_1$) βρίσκοντας επιτυχώς το ελάχιστο.
Οι 3 πρώτες προβολές δεν ανήκουν μέσα στο $X$ και για αυτό επιλέγονται οριακές συντεταγμένες.

Σε σχέση με το (a,i) ο αλγόριθμος συγκλίνει γρηγορότερα καθώς σε κάθε βήμα το $x_k$ μειώνεται με μεγαλύτερο ρυθμό.
Δηλαδή, στο (a,i) είχαμε $x_{k+1} = 0.9 \cdot x_k$ ενώ εδώ έχουμε $x_{k+1} = 0.5 \cdot x_k$ (όταν η προβολή είναι μέσα στο $X$).

Στο (a,iv) είδαμε τον λόγο που ο αλγόριθμος χωρίς προβολή δεν συγκλίνει προς το ελάχιστο. Στην περίπτωση με την προβολή συγκλίνει καθώς ικανοποιείται η συνθήκη \ref{eq:s-gamma-requirements}.

\begin{figure}[htbp]
	\centerfloat
	\includegraphics{plots/algorithms/plot_b.eps}
	\caption{Σύγκλιση της μεγίστης καθόδου με προβολή για $s_k = 15$, $\gamma_k = 0.1$, $\epsilon = 0.01$}
%	\label{fig:untitled}
\end{figure}

%\begin{figure}[htp]
%	\centerfloat
%	\setlength\figureheight{\textwidth}
%	\setlength\figurewidth{\textwidth}
%	\includetikz{plots/algorithms/plot_b}
%	\caption{Σύγκλιση της μεγίστης καθόδου με προβολή για 2 }
%%	\label{fig:untitled}
%\end{figure}

\chapter{Ερώτημα (γ)}
\subsection{Ερώτημα (δ)}


Χρησιμοποιούμε τα ορίσματα $s_k = 0.1$, $\gamma_k = 0.01$, $\epsilon = 0.01$ και αρχικό σημείο το $(11,3)$.
Οι σχέσεις \ref{eq:x1-final} και \ref{eq:x2-final} γίνονται:
\begin{align}
	\label{eq:x1-finald}
	x_{1_{k+1}} =		
		\begin{cases}
			0.99 \cdot x_{1_{k}} - 0.2, &$ αν $ x_{1_{k}} \leq -\frac{200}{9} \\
			0.999 \cdot x_{1_{k}}, &$ αν $ -\frac{200}{9} < x_{1_{k}} < \frac{100}{9}\\
			0.99 \cdot x_{1_{k}} + 0.1, &$ αν $ x_{1_{k}} \geq \frac{100}{9}
		\end{cases}
\end{align}
και
\begin{align}
	\label{eq:x2-finald}
	x_{2_{k+1}} = 	
		\begin{cases}
			0.99 \cdot x_{2_{k}} - 0.12, &$ αν $ x_{2_{k}} \leq -\frac{40}{3} \\
			0.999 \cdot x_{2_{k}}, &$ αν $ -\frac{40}{3} < x_{2_{k}} < \frac{50}{3} \\
			0.99 \cdot x_{2_{k}} + 0.15, &$ αν $ x_{2_{k}} \geq \frac{50}{3}
		\end{cases}
\end{align}

Παρατηρούμε ότι άν το $0.9 \cdot x_k \in X$ τότε $x_{k+1} = 0.999 \cdot x_k$. 
Άρα αν $0.9 \cdot x_k \in X$ τότε $0.9 \cdot x_{k+1} \in X$. 
Έτσι, αν για κάποιο $k$ έχουμε $x_k \in X$ τότε $\forall k' > k$ θα ισχύει $x_{k'} \in X$.
Τα $s$ και $\gamma$ ικανοποιούν την σχέση \ref{eq:s-gamma-requirements}

Μάλιστα, για το αρχικό σημείο $x_1 = (11,3)$ ισχύει $\overbar{x_1} = \Pr\{0.9 \cdot x_1\}$. Όμως $0.9 \cdot x_1 = (9.9, 2.7) \in X$ και άρα $\overbar{x_1} = 0.999 \cdot x_k = (10.989, 2.997)$.
Δηλαδή, θα ισχύει $\overbar{x_k} = 0.999 \cdot x_k \forall k$.
Έτσι, εκ των προτέρων, μπορούμε να καταλάβουμε ότι ο αλγόριθμος θα συγκλίνει και να υπολογίσουμε τον αριθμό των βημάτων.

Έχουμε
\begin{gather}
	x_{k+1} = (1 - \gamma) \cdot x_k + \gamma \cdot (1-s) \cdot x_k =\nonumber\\
	x_k \cdot (1 - s \cdot \gamma)
\end{gather}

\begin{varwidth}{\textwidth}
Οπότε, για αρχικό σημείο $x_1$ η ακολουθία $\{x_k\}$ έχει ως εξής:
\begin{itemize}
	\item $x_2 = (1 - s \cdot \gamma) \cdot x_1$
	\item $x_3 = (1 - s \cdot \gamma) \cdot x_2 = (1 - s \cdot \gamma)^2 \cdot x_1$
	\item $x_4 = (1 - s \cdot \gamma) \cdot x_3 = (1 - s \cdot \gamma)^3 \cdot x_1$
	\item \ldots
	\item $x_k = (1 - s \cdot \gamma)^{k-1} \cdot x_1$
\end{itemize}
\end{varwidth}
\hfill \break
\hfill \break

Ισχύει
\begin{gather*}
	\norm{x_{k+1} - \overbar{x_k}} = \\
	\norm{(1-\gamma) \cdot x_k + \gamma \cdot \overbar{x_k} - \overbar{x_k}} = \\
	\norm{ (1-\gamma) \cdot x_k + (\gamma - 1) \cdot \overbar{x_k} } = \\
	\norm{ (1-\gamma) - (1 - \gamma) \cdot (1 -s) \cdot x_k} = \\
	\norm{ s \cdot (1-\gamma) \cdot x_k } = \\
	\abs{s \cdot (1-\gamma) \cdot (1 - s \cdot \gamma)^{k-1}} \cdot \norm{x_1} 
\end{gather*}

Από την συνθήκη τερματισμού \ref{eq:termination-condition} και αντικαθιστώντας τις τιμές για $\epsilon$, $\gamma$, $s$:
\begin{gather*}
	\abs{s \cdot (1-\gamma) \cdot (1 - s \cdot \gamma)^{k-1}} \cdot \norm{x_1} = \epsilon \implies\\
	(1 - s \cdot \gamma) ^ {k - 1} = \frac{\epsilon}{s \cdot \abs{1-\gamma} \cdot \norm{x_1}} \implies\\
	k = \frac{\ln{\frac{\epsilon}{s \cdot \abs{1-\gamma} \cdot \norm{x_1}}}}{\ln{\abs{1 - s \cdot \gamma}}} + 1 \implies\\
	k =  4725.938 \implies\\
	k = 4726
\end{gather*}

Αν εκτελέσουμε τον αλγόριθμο επιβεβαιώνονται αυτά τα αποτελέσματα αφού ο αυτός τερματίζει για $k = 4726$. Αυτή η ταχύτητα σύγκλισης είναι σχετικά αργή και για να την βελτιώσουμε θα πρέπει να αυξήσουμε τις τιμές του $\gamma$ και $s$ οι οποίες όμως πρέπει να τηρούν την σχέση \ref{eq:s-gamma-requirements}.

\begin{figure}[htbp]
	\centerfloat
	\includegraphics{plots/algorithms/plot_d.eps}
	\caption{Σύγκλιση της μεγίστης καθόδου με προβολή για $s_k = 0.1$, $\gamma_k = 0.01$, $\epsilon = 0.01$}
\end{figure}


\end{document}
