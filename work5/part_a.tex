% part_a
\chapter{Θεωρητική λύση με χρήση του θεωρήματος Karush–Kuhn–Tucker} \label{part_a}

\section{Συνάρτηση f}

Για το πρόβλημα
\begin{equation}
	\label{eq:f-problem}
	\min_{
		\substack{3 \leq x_1 \leq 30 \\\\ -25 \leq x_2 \leq -5}
		}
	f(x_1, x_2)
\end{equation}
Οι περιορισμοί γράφονται ως εξής:
\begin{align*}
	g_{1_f} = 3 - x_1 &\leq 0\\
	g_{2_f} = x_1 - 30 &\leq 0\\
	g_{3_f} = -x_2 -25 &\leq 0\\
	g_{4_f} = x_2 + 5 &\leq 0
\end{align*}
και οι κλίσεις τους υπολογίζονται ως:
\begin{align*}
	\nabla g_{1_f} =& 
		\begin{bmatrix}
			-1& 0
		\end{bmatrix}^T\\
	\nabla g_{2_f} =& 
		\begin{bmatrix}
			1& 0
		\end{bmatrix}^T\\
	\nabla g_{3_f} =& 
		\begin{bmatrix}
			0& -1
		\end{bmatrix}^T\\
	\nabla g_{4_f} =& 
		\begin{bmatrix}
			0& 1
		\end{bmatrix}^T\\
\end{align*}
Τα κριτήρια του Karush-Kuhn-Tucker γράφονται ως εξής:
\begin{enumerate}[i)]
	\item $\lambda_1 \geq 0$, $\lambda_2 \geq 0$, $\lambda_3 \geq 0$, $\lambda_4 \geq 0$
	\item \begin{align*}
			\lambda_1 \cdot (x_1 - 3) = 0\\
			\lambda_2 \cdot (x_1 - 30) = 0\\
			\lambda_3 \cdot (x_2 + 25) = 0\\
			\lambda_4 \cdot (x_2 + 5) = 0
		\end{align*}
	\item \begin{equation*}
		\begin{bmatrix}
			4 \cdot x_1 - 3 \cdot x_2\\
			-3 \cdot x_1 + 4 \cdot x_2
		\end{bmatrix} + 
		\begin{bmatrix}
			-\lambda_1\\
			0
		\end{bmatrix} +
		\begin{bmatrix}
			\lambda_2\\
			0
		\end{bmatrix} +
		\begin{bmatrix}
			0\\
			-\lambda_3
		\end{bmatrix} +		
		\begin{bmatrix}
			0\\
			\lambda_4
		\end{bmatrix} = 
		\begin{bmatrix}
			0\\
			0
		\end{bmatrix}
	\end{equation*}
\end{enumerate}
Λύνοντας τα παραπάνω συστήματα παίρνουμε τις λύσεις
\begin{equation}
	(x_1^*, x_2^*) = (3, -5)
\end{equation}
με διάνυσμα πολλαπλασιαστών Lagrange:
\begin{equation*}
	(\lambda_1^*, \lambda_2^*, \lambda_3^*, \lambda_4^*) = (27, 0, 0, 29)
\end{equation*}

\newpage
\section{Συνάρτηση g}

Για το πρόβλημα
\begin{equation}
	\label{eq:g-problem}
	\min_{
		\substack{x_1 \leq -1 \\ x_2 \leq -1}
		}
	g(x_1, x_2)
\end{equation}
Οι περιορισμοί γράφονται ως εξής:
\begin{align*}
	g_{1_f} = x_1 + 1 &\leq 0\\
	g_{2_f} = x_2 + 1 &\leq 0
\end{align*}
και οι κλίσεις τους υπολογίζονται ως:
\begin{align*}
	\nabla g_{1_f} =& 
		\begin{bmatrix}
			1& 0
		\end{bmatrix}^T\\
	\nabla g_{2_f} =& 
		\begin{bmatrix}
			0& 1
		\end{bmatrix}^T
\end{align*}
Τα κριτήρια του Karush-Kuhn-Tucker γράφονται ως εξής:
\begin{enumerate}[i)]
	\item $\lambda_1 \geq 0$, $\lambda_2 \geq 0$
	\item \begin{align*}
			\lambda_1 \cdot (x_1 + 1) = 0\\
			\lambda_2 \cdot (x_2 + 1) = 0
		\end{align*}
	\item \begin{equation*}
		\begin{bmatrix}
			2 \cdot x_1 - 2 \cdot x_2\\
			-2 \cdot x_1 + 2 \cdot x_2
		\end{bmatrix} + 
		\begin{bmatrix}
			\lambda_1\\
			0
		\end{bmatrix} +
		\begin{bmatrix}
			0\\
			\lambda_2
		\end{bmatrix} = 
		\begin{bmatrix}
			0\\
			0
		\end{bmatrix}
	\end{equation*}
\end{enumerate}

Η λύση αυτού του συστήματος προκύπτει
\begin{equation}
	(x_1^*, x_2^*) = (t, t) \qquad t \leq -1
\end{equation}
Δηλαδή, το πρόβλημα \ref{eq:g-problem} δεν έχει μοναδική λύση καθώς η $g(x_1, x_2)$ δεν είναι γνησίως κυρτή.