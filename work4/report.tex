%&preamble
\usepackage{varwidth}
%% end of static part. Save at preamble.tex and use command: xelatex -ini -shell-escape -job-name="preamble" "&xelatex preamble.tex\dump"
%% dynamic part or stuff that can't be in the precompiled file
\usetikzlibrary{external}
\tikzset{external/system call={xelatex -fmt=preamble.fmt \tikzexternalcheckshellescape -halt-on-error -interaction=batchmode -jobname "\image" "\texsource"}}
\tikzexternalize
%delete external pdf if old
\newcommand{\deloldext}[2]{%  
   \immediate\write18{./deloldext.pl '#1' '#2'}
}
%include tikz file and call \deloldext
\newcommand{\includetikz}[1]{%
    \tikzsetnextfilename{#1}%
    \deloldext{#1.tikz}{#1.pdf}%
    \input{#1.tikz}%
}

% Only for xelatex and lualatex. It provides an automatic and unified interface to feature-rich AAT and OpenType fonts.
% https://ctan.org/pkg/fontspec
\usepackage{fontspec}
\setmainfont{DejaVu Serif}
% Change the title of contents list and figure list
\renewcommand{\contentsname}{Περιεχόμενα}
\renewcommand{\listfigurename}{Λίστα Σχημάτων}
% Change the name of each figure
\renewcommand{\figurename}{Σχήμα}
\renewcommand{\lstlistingname}{Καταχώρηση}% Listing -> Algorithm
\renewcommand{\lstlistlistingname}{List of \lstlistingname s}


% defines a variable
%\def \variableName {Something that's better to use as a variable}

% \\ or \newline == newline, no paragraph
% \\* == no page break

\title{4η Εργαστηριακή Άσκηση}
\author{Ορέστης Φλώρος-Μαλιβίτσης\\
  Τομέας Ηλεκτρονικής,\\
  Τμήμα Ηλ. Μηχανικών / Μηχανικών ΗΥ,\\
  Αριστοτέλειο Πανεπιστήμιο Θεσσαλονίκης}
\date{22/05/2015}

%\includeonly{projection_intro, part_b, part_d} % delete for release version
\newcommand{\fakealign}{%
   \mbox{\hspace{5cm}} & \mbox{\hspace{5cm}} \nonumber\\%
}
% Quickly check the document. To produce pages, comment it out
%\usepackage{syntonly}
%\syntaxonly

\begin{document}
\maketitle
\tableofcontents
\listoffigures
\newpage


% Δομή του Project
\chapter*{Δομή του Project} \label{project-structure}

\begin{description}
	\item[Floros\_orestis\_Work4\_7796.pdf] Αυτή η αναφορά.
	\item[work4.pdf] Η εκφώνηση της εργαστηριακής άσκησης.
	\item[plots] Φάκελος που βρίσκονται οι διάφορες γραφικές παραστάσεις σε μορφή png. Επίσης, περιέχει τους κώδικες Matlab για την δημιουργία αυτών των γραφικών παραστάσεων.
	\item[functions] Φάκελος με την συνάρτηση και την παράγωγό της.
\end{description}

% Εισαγωγή
\chapter{Εισαγωγή} \label{intro}

Δίνονται οι συναρτήσεις για ελαχιστοποίηση:
\begin{equation}
	\label{eq:f-definition}
	f:\mathbb{R}^{2} \mapsto \mathbb{R}, \qquad f(x_1, x_2) = x_1 \cdot x_2 + 2 \cdot (x_1 - x_2)^2
\end{equation}
με περιορισμούς
\begin{align}
	3 \leq x_1& \leq 30\\
	-25 \leq x_2& \leq -50
\end{align}
και
\begin{equation}
	\label{eq:g-definition}
	g:\mathbb{R}^{2} \mapsto \mathbb{R}, \qquad g(x_1, x_2) = (x_1 - x_2)^2
\end{equation}
με περιορισμούς
\begin{align}
	x_1& \leq -1\\
	x_2& \leq -1
\end{align}
Υπολογίζονται οι κλίσεις τους:

\begin{equation}
\nabla f(x_1, x_2) = 
\begin{bmatrix}
	4 \cdot x_1 - 3 \cdot x_2\\
	-3 \cdot x_1 + 4 \cdot x_2
\end{bmatrix}
\end{equation}
και
\begin{equation}
\nabla g(x_1, x_2) = 
\begin{bmatrix}
	2 \cdot (x_1 - x_2)\\
	-2 \cdot (x_1 - x_2)
\end{bmatrix}
\end{equation}
Οι Εσσιανοί πίνακες υπολογίζονται ως:
\begin{equation}
\nabla^2 f(x_1, x_2) = 
\begin{bmatrix}
	4 & -3\\
	-3 & 4
\end{bmatrix}
\end{equation} με ιδιοτιμές $\lambda_1 = 7$ και $\lambda_2 = 1$.
\begin{equation}
\nabla^2 f(x_1, x_2) = 
\begin{bmatrix}
	2 & -2\\
	-2 & 2
\end{bmatrix}
\end{equation} με ιδιοτιμές $\lambda_1 = 4$ και $\lambda_2 = 0$.\\
Παρατηρούμε ότι λόγω των Εσσιανών τους πινάκων οι $f$ και $g$ είναι κυρτές συναρτήσεις και τα αντίστοιχα προβλήματα είναι κυρτά και υπερ-αποτελούμενα.\\
Παρουσιάζονται σχετικές γραφικές παραστάσεις.

\setlength{\LTleft}{-20cm plus -1fill}
\setlength{\LTright}{\LTleft}

\section{Συνάρτηση f}
\begin{longtable}{cc}
	\centering
	\includegraphics[width=100mm]{plots/functions/f_mesh_plot.eps}&
	\includegraphics[width=100mm]{plots/functions/f_mesh_plot_limited.eps}\\
	\includegraphics[width=100mm]{plots/functions/f_contourf_plot.eps} &
	\includegraphics[width=100mm]{plots/functions/f_contourf_plot_limited.eps}\\
	\includegraphics[width=100mm]{plots/functions/gradf_mesh_plot.eps} &
	\includegraphics[width=100mm]{plots/functions/gradf_mesh_plot_limited.eps}\\
\end{longtable}

\newpage
\section{Συνάρτηση g}
\setlength{\LTleft}{-20cm plus -1fill}
\setlength{\LTright}{\LTleft}
\begin{longtable}{cc}
	\centerfloat
	\includegraphics[width=100mm]{plots/functions/g_mesh_plot.eps} &
	\includegraphics[width=100mm]{plots/functions/g_mesh_plot_limited.eps}\\
	\includegraphics[width=100mm]{plots/functions/g_contourf_plot.eps} &
	\includegraphics[width=100mm]{plots/functions/g_contourf_plot_limited.eps}\\
	\includegraphics[width=100mm]{plots/functions/gradg_mesh_plot.eps} &
	\includegraphics[width=100mm]{plots/functions/gradg_mesh_plot_limited.eps}\\
\end{longtable}
% part_a
\chapter{Θεωρητική λύση με χρήση του θεωρήματος Karush–Kuhn–Tucker} \label{part_a}

\section{Συνάρτηση f}

Για το πρόβλημα
\begin{equation}
	\label{eq:f-problem}
	\min_{
		\substack{3 \leq x_1 \leq 30 \\\\ -25 \leq x_2 \leq -5}
		}
	f(x_1, x_2)
\end{equation}
Οι περιορισμοί γράφονται ως εξής:
\begin{align*}
	g_{1_f} = 3 - x_1 &\leq 0\\
	g_{2_f} = x_1 - 30 &\leq 0\\
	g_{3_f} = -x_2 -25 &\leq 0\\
	g_{4_f} = x_2 + 5 &\leq 0
\end{align*}
και οι κλίσεις τους υπολογίζονται ως:
\begin{align*}
	\nabla g_{1_f} =& 
		\begin{bmatrix}
			-1& 0
		\end{bmatrix}^T\\
	\nabla g_{2_f} =& 
		\begin{bmatrix}
			1& 0
		\end{bmatrix}^T\\
	\nabla g_{3_f} =& 
		\begin{bmatrix}
			0& -1
		\end{bmatrix}^T\\
	\nabla g_{4_f} =& 
		\begin{bmatrix}
			0& 1
		\end{bmatrix}^T\\
\end{align*}
Τα κριτήρια του Karush-Kuhn-Tucker γράφονται ως εξής:
\begin{enumerate}[i)]
	\item $\lambda_1 \geq 0$, $\lambda_2 \geq 0$, $\lambda_3 \geq 0$, $\lambda_4 \geq 0$
	\item \begin{align*}
			\lambda_1 \cdot (x_1 - 3) = 0\\
			\lambda_2 \cdot (x_1 - 30) = 0\\
			\lambda_3 \cdot (x_2 + 25) = 0\\
			\lambda_4 \cdot (x_2 + 5) = 0
		\end{align*}
	\item \begin{equation*}
		\begin{bmatrix}
			4 \cdot x_1 - 3 \cdot x_2\\
			-3 \cdot x_1 + 4 \cdot x_2
		\end{bmatrix} + 
		\begin{bmatrix}
			-\lambda_1\\
			0
		\end{bmatrix} +
		\begin{bmatrix}
			\lambda_2\\
			0
		\end{bmatrix} +
		\begin{bmatrix}
			0\\
			-\lambda_3
		\end{bmatrix} +		
		\begin{bmatrix}
			0\\
			\lambda_4
		\end{bmatrix} = 
		\begin{bmatrix}
			0\\
			0
		\end{bmatrix}
	\end{equation*}
\end{enumerate}
Λύνοντας τα παραπάνω συστήματα παίρνουμε τις λύσεις
\begin{equation}
	(x_1^*, x_2^*) = (3, -5)
\end{equation}
με διάνυσμα πολλαπλασιαστών Lagrange:
\begin{equation*}
	(\lambda_1^*, \lambda_2^*, \lambda_3^*, \lambda_4^*) = (27, 0, 0, 29)
\end{equation*}

\newpage
\section{Συνάρτηση g}

Για το πρόβλημα
\begin{equation}
	\label{eq:g-problem}
	\min_{
		\substack{x_1 \leq -1 \\ x_2 \leq -1}
		}
	g(x_1, x_2)
\end{equation}
Οι περιορισμοί γράφονται ως εξής:
\begin{align*}
	g_{1_f} = x_1 + 1 &\leq 0\\
	g_{2_f} = x_2 + 1 &\leq 0
\end{align*}
και οι κλίσεις τους υπολογίζονται ως:
\begin{align*}
	\nabla g_{1_f} =& 
		\begin{bmatrix}
			1& 0
		\end{bmatrix}^T\\
	\nabla g_{2_f} =& 
		\begin{bmatrix}
			0& 1
		\end{bmatrix}^T
\end{align*}
Τα κριτήρια του Karush-Kuhn-Tucker γράφονται ως εξής:
\begin{enumerate}[i)]
	\item $\lambda_1 \geq 0$, $\lambda_2 \geq 0$
	\item \begin{align*}
			\lambda_1 \cdot (x_1 + 1) = 0\\
			\lambda_2 \cdot (x_2 + 1) = 0
		\end{align*}
	\item \begin{equation*}
		\begin{bmatrix}
			2 \cdot x_1 - 2 \cdot x_2\\
			-2 \cdot x_1 + 2 \cdot x_2
		\end{bmatrix} + 
		\begin{bmatrix}
			\lambda_1\\
			0
		\end{bmatrix} +
		\begin{bmatrix}
			0\\
			\lambda_2
		\end{bmatrix} = 
		\begin{bmatrix}
			0\\
			0
		\end{bmatrix}
	\end{equation*}
\end{enumerate}

Η λύση αυτού του συστήματος προκύπτει
\begin{equation}
	(x_1^*, x_2^*) = (t, t) \qquad t \leq -1
\end{equation}
Δηλαδή, το πρόβλημα \ref{eq:g-problem} δεν έχει μοναδική λύση καθώς η $g(x_1, x_2)$ δεν είναι γνησίως κυρτή.
\chapter{Μέθοδος της μέγιστης καθόδου με προβολή}

\section{Θεωρητική ανάλυση}

Θεωρούμε τους περιορισμούς \[-20 \leq x_1 \leq 10\] και \[-12 \leq x_2 \leq 15\] και θα χρησιμοποιήσουμε τον αλγόριθμο της μέγιστης καθόδου με προβολή για να βρούμε το ελάχιστο. 

Η ακολουθία των σημείων $x_k$ αυτής της μεθόδου δίνεται από την σχέση 
\begin{equation}
	\label{eq:x-original}
	x_{k+1} = x_k + \gamma_k \cdot (\overbar{x_k} - x_k) 
\end{equation}
ή
\begin{equation}
	\label{eq:x-original2}
	x_{k+1} = (1-\gamma) \cdot x_k + \gamma \cdot \overbar{x_k}
\end{equation}
όπου
\begin{equation}
	\label{eq:xbar-original} 
	\overbar{x_k} = \Pr\{x_k - s_k \cdot \nabla f(x_k)\} 
\end{equation}
Αντικαθιστώντας την κλίση \ref{eq:f-nabla} στην \ref{eq:xbar-original} παίρνουμε:
\begin{equation*}
	\overbar{x_k} = \Pr\{x_k - s \cdot x_k\}
\end{equation*}
και τελικά
\begin{equation}
	\overbar{x_k} = \Pr\{(1-s) \cdot x_k\}
\end{equation}

Στο πρόβλημά μας οι περιορισμοί εκφράζονται από φράγματα στις μεταβλητές της $f(x)$. Το σύνολο των εφικτών σημείων, 
\[X = \{(x_1, x_2) \in \mathbb{R}^n: -20 \leq x_1 \leq 10, -12 \leq x_2 \leq 15 \}\]
είναι κυρτό και η προβολή για κάθε σημείο $x_k = (x_1, x_2)$ δίνεται από την σχέση στη σελίδα 202 του βιβλίου:
\begin{align}
	\label{eq:pr1-in-X}
	[\Pr_X\{x_1\}]_1 =		
		\begin{cases}
			-20, &$ αν $ x_1 \leq -20 \\
			x_1, &$ αν $ -20 < x_1 < 10 \\
			10, &$ αν $ x_1 \geq 10
		\end{cases}
\end{align}

\begin{align}
	\label{eq:pr2-in-X}
	[\Pr_X\{x_2\}]_2 =		
		\begin{cases}
			-12, &$ αν $ x_2 \leq -12 \\
			x_2, &$ αν $ -12 < x_2 < 15 \\
			15, &$ αν $ x_2 \geq 15
		\end{cases}
\end{align}

Αντικαθιστώντας στις σχέσεις \ref{eq:pr1-in-X} και \ref{eq:pr2-in-X} για το σημείο $(1-s) \cdot x_k$ έχουμε:
\begin{align}
	\label{eq:pr1-in-X-replaced}
	[\Pr_X\{(1-s) \cdot x_1 \}]_1 =		
		\begin{cases}
			-20, &$ αν $ (1-s) \cdot x_1 \leq -20 \\
			(1-s) \cdot x_1, &$ αν $ -20 < (1-s) \cdot x_1 < 10 \\
			10, &$ αν $ (1-s) \cdot x_1 \geq 10
		\end{cases}
\end{align}
\begin{align}
	\label{eq:pr2-in-X-replaced}
	[\Pr_X\{(1-s) \cdot x_2\}]_2 =		
		\begin{cases}
			-12, &$ αν $ (1-s) \cdot x_2 \leq -12 \\
			(1-s) \cdot x_2, &$ αν $ -12 < (1-s) \cdot x_2 < 15 \\
			15, &$ αν $ (1-s) \cdot x_2 \geq 15
		\end{cases}
\end{align}

Παρατηρούμε ότι για $s = 1$ έχουμε $\overbar{x_k} = \Pr\{(1-s) \cdot x_k\} = \Pr\{(0,0)\} = (0,0)$, δηλαδή η μέθοδος είναι ίδια με την μέθοδο της μεγίστης καθόδου χωρίς προβολή.

Αντικαθιστώντας την \ref{eq:pr1-in-X-replaced}, \ref{eq:pr2-in-X-replaced} και \ref{eq:xbar-original} στην \ref{eq:x-original2} παίρνουμε:
\begin{align}
	\label{eq:x1-final}
	x_{1_{k+1}} =	(1-\gamma) \cdot x_{1_{k}} + 	
		\begin{cases}
			-20 \cdot \gamma, &$ αν $ (1-s) \cdot x_{1_{k}} \leq -20 \\
			(1-s) \cdot \gamma \cdot x_{1_{k}}, &$ αν $ -20 < (1-s) \cdot x_{1_{k}} < 10 \\
			10 \cdot \gamma, &$ αν $ (1-s) \cdot x_{1_{k}} \geq 10
		\end{cases}
\end{align}
και
\begin{align}
	\label{eq:x2-final}
	x_{2_{k+1}} =	(1-\gamma) \cdot x_{2_{k}} + 	
		\begin{cases}
			-12 \cdot \gamma, &$ αν $ (1-s) \cdot x_{2_{k}} \leq -12 \\
			(1-s) \cdot \gamma \cdot x_{2_{k}}, &$ αν $ -12 < (1-s) \cdot x_{2_{k}} < 15 \\
			15 \cdot \gamma, &$ αν $ (1-s) \cdot x_{2_{k}} \geq 15
		\end{cases}
\end{align}
Αν πάρουμε τα $x_{1_k}, x_{2_k}$ ώστε να ικανοποιούν τους περιορισμούς ισχύει
\[x_{k+1} = (1 - s \cdot \gamma) \cdot x_{k} \]
είναι προφανές ότι για να έχουμε και πάλι σύγκλιση προς το ελάχιστο αφού βρεθούμε σε κάποιο εφικτό σημείο $x_k$ θέλουμε να ισχύει
\[\abs{1 - s \cdot \gamma} < 1 \]
δηλαδή
\begin{equation} 
\label{eq:s-gamma-requirements}
0 < s \cdot \gamma < 2
 \end{equation}

\section{Προγραμματιστική υλοποίηση}

Ο αλγόριθμος υλοποιείται στο αρχείο \hyperref{steepest/projection_algorithm.m}{categoryname}{labelname}{projection\_algorithm.m}. Δίνεται το βήμα της κυρίως επανάληψης:


\begin{lstlisting}[language=Matlab, escapechar=|, caption=Κυρίως βήμα στην υλοποίηση της μεθόδου μεγίστης καθόδου με προβολή]
% main loop
while true
    g = gradf(x(1, k) ,x(2, k));
    
    % terminate by comparing the norm(gradf) with e.
    if norm_terminate
        if norm(g) < e	| \label{line:norm-termination} |
            break
        end
    end
    
    % the point that will be projected
    selected(:, k) = x(:, k) - s * g;
    % calculating xbar
    xbar(:, k) = ...
        a .* (selected(:,k) <= a) + ...  % smaller than low bound
        b .* (selected(:, k) >= b) + ... % greater than uper bound
        selected(:, k) .* (a < selected(:, k) ) .* (selected(:, k) < b); % inside bounds
    d = (xbar(:, k) - x(:, k));
    % calculating the next point
    x(:, k + 1) = x(:, k) + gamma * d;
    
    % terminate by comparing the norm of x_{k+1} - xbar_k with e
    if ~norm_terminate
        if norm(x(:, k+1) - xbar(:, k)) < e | \label{line:xbar-termination} |
            break
        end
    end    
    
    k = k + 1;
    if (k > max_k) 
        break; 
    end
end
\end{lstlisting}

Για τον τερματισμό του αλγορίθμου δίνονται δύο επιλογές: τερματισμός με τον έλεγχο του μεγέθους της νόρμας $\nabla f(x_k) < \epsilon$ (γραμμή \ref{line:norm-termination}) και με τον έλεγχο που δίνεται στην σελίδα 201 του βιβλίου (γραμμή \ref{line:xbar-termination}). Στην δεύτερη περίπτωση ο αλγόριθμος τερματίζεται όταν:
\begin{equation}
\label{eq:termination-condition}
\norm{x_{k+1} - \overbar{x_k}} < \epsilon
\end{equation}
Εκτός και αν αναφέρεται κάτι άλλο θα χρησιμοποιούμε τον δεύτερο τρόπο τερματισμού.

\subsection{Ερώτημα (β)}


Χρησιμοποιούμε τα ορίσματα $s_k = 15$, $\gamma_k = 0.1$, $\epsilon = 0.01$ και αρχικό σημείο το $(8,3)$.
Οι σχέσεις \ref{eq:x1-final} και \ref{eq:x2-final} γίνονται:
\begin{align}
	\label{eq:x1-finalb}
	x_{1_{k+1}} =		
		\begin{cases}
			0.9 \cdot x_{1_{k}} - 2, &$ αν $ x_{1_{k}} \geq \frac{10}{7} \\
			-0.5 \cdot x_{1_{k}}, &$ αν $ -\frac{5}{7} < x_{1_{k}} < \frac{10}{7}\\
			0.9 \cdot x_{1_{k}} + 1 \cdot, &$ αν $ x_{1_{k}} \leq -\frac{5}{7}
		\end{cases}
\end{align}
και
\begin{align}
	\label{eq:x2-finalb}
	x_{2_{k+1}} = 	
		\begin{cases}
			0.9 \cdot x_{2_{k}} - 1.2, &$ αν $ x_{2_{k}} \geq \frac{6}{7} \\
			-0.5 \cdot x_{2_{k}}, &$ αν $ -\frac{15}{14} < x_{2_{k}} < \frac{6}{7} \\
			0.9 \cdot x_{2_{k}} + 1.5 \cdot, &$ αν $ x_{2_{k}} \leq -\frac{15}{14}
		\end{cases}
\end{align}

Παρατηρούμε ότι άν το $-14 \cdot x_k \in X$ τότε $x_{k+1} = -0.5 \cdot x_k$. 
Για τους συγκεκριμένους περιορισμούς για οποιοδήποτε $x_k \in X$ ισχύει $-0.5 \cdot x_k \in X$ άρα αν $-14 \cdot x_k \in X$ τότε $-14 \cdot x_{k+1} \in X$. 
Έτσι, αν για κάποιο $k$ έχουμε $x_k \in X$ τότε $\forall k' > k$ θα ισχύει $x_{k'} \in X$.

Τα $s$ και $\gamma$ ικανοποιούν την σχέση \ref{eq:s-gamma-requirements}.
Όταν εκτελούμε τον αλγόριθμο τερματίζεται για $k = 15$ (για αρχικό σημείο $x_1$) βρίσκοντας επιτυχώς το ελάχιστο.
Οι 3 πρώτες προβολές δεν ανήκουν μέσα στο $X$ και για αυτό επιλέγονται οριακές συντεταγμένες.

Σε σχέση με το (a,i) ο αλγόριθμος συγκλίνει γρηγορότερα καθώς σε κάθε βήμα το $x_k$ μειώνεται με μεγαλύτερο ρυθμό.
Δηλαδή, στο (a,i) είχαμε $x_{k+1} = 0.9 \cdot x_k$ ενώ εδώ έχουμε $x_{k+1} = 0.5 \cdot x_k$ (όταν η προβολή είναι μέσα στο $X$).

Στο (a,iv) είδαμε τον λόγο που ο αλγόριθμος χωρίς προβολή δεν συγκλίνει προς το ελάχιστο. Στην περίπτωση με την προβολή συγκλίνει καθώς ικανοποιείται η συνθήκη \ref{eq:s-gamma-requirements}.

\begin{figure}[htbp]
	\centerfloat
	\includegraphics{plots/algorithms/plot_b.eps}
	\caption{Σύγκλιση της μεγίστης καθόδου με προβολή για $s_k = 15$, $\gamma_k = 0.1$, $\epsilon = 0.01$}
%	\label{fig:untitled}
\end{figure}

%\begin{figure}[htp]
%	\centerfloat
%	\setlength\figureheight{\textwidth}
%	\setlength\figurewidth{\textwidth}
%	\includetikz{plots/algorithms/plot_b}
%	\caption{Σύγκλιση της μεγίστης καθόδου με προβολή για 2 }
%%	\label{fig:untitled}
%\end{figure}

\subsection{Ερώτημα (γ)}

Χρησιμοποιούμε τα ορίσματα $s_k = 20$, $\gamma_k = 0.3$, $\epsilon = 0.02$ και αρχικό σημείο το $(-5,7)$.
Οι σχέσεις \ref{eq:x1-final} και \ref{eq:x2-final} γίνονται:
\begin{align}
	\label{eq:x1-finalc}
	x_{1_{k+1}} =		
		\begin{cases}
			0.7 \cdot x_{1_{k}} - 6, &$ αν $ x_{1_{k}} \geq \frac{20}{19} \\
			-5 \cdot x_{1_{k}}, &$ αν $ -\frac{10}{19} < x_{1_{k}} < \frac{20}{19}\\
			0.7 \cdot x_{1_{k}} + 3 \cdot, &$ αν $ x_{1_{k}} \leq -\frac{10}{19}
		\end{cases}
\end{align}
και
\begin{align}
	\label{eq:x2-finalc}
	x_{2_{k+1}} = 	
		\begin{cases}
			0.7 \cdot x_{2_{k}} - 3.6, &$ αν $ x_{2_{k}} \geq \frac{12}{19} \\
			-5 \cdot x_{2_{k}}, &$ αν $ -\frac{15}{19} < x_{2_{k}} < \frac{12}{19} \\
			0.7 \cdot x_{2_{k}} + 4.5 \cdot, &$ αν $ x_{2_{k}} \leq -\frac{15}{19}
		\end{cases}
\end{align}

Παρατηρούμε ότι άν το $-19 \cdot x_k \in X$ τότε $x_{k+1} = -5 \cdot x_k$. 
Aν $-19 \cdot x_k \in X$ δεν ισχύει αναγκαστικά $-19 \cdot x_{k+1} \in X$. 
Τα $s$ και $\gamma$ δεν ικανοποιούν την σχέση \ref{eq:s-gamma-requirements} και για αυτόν τον λόγο δεν περιμένουμε να συγκλίνει. Όποτε το $\overbar{x_k}$ προβάλλεται μέσα στο $X$, το $x_{k+1}$ θα είναι πιο μακριά από την αρχή τον αξόνων σε σχέση με το $x_k$.

Πράγματι, όταν εκτελούμε τον αλγόριθμο αυτός δεν τερματίζεται πριν βρεθεί το ελάχιστο αλλά αφού φτάσουμε στον μέγιστο αριθμό βημάτων.

Σε σχέση με το (a,i) ο αλγόριθμος δεν συγκλίνει εξαιτίας των λόγων που περιγράψαμε.

Στην περίπτωση (a,iv) ο αλγόριθμος δεν συγκλίνει και απειρίζεται. Στην περίπτωση (γ) ο αλγόριθμος δεν συγκλίνει αλλά δεν απειρίζεται. Αυτό εξηγείται επειδή το $\overbar{x_k}$ μπορεί να πάρει περιορισμένες τιμές. Αυτό φαίνεται και στο γράφημα \ref{fig:c_xbar}

\begin{figure}[htbp]
	\centerfloat
	\includegraphics{plots/algorithms/plot_c.eps}
	\caption{Σύγκλιση της μεγίστης καθόδου με προβολή για $s_k = 20$, $\gamma_k = 0.3$, $\epsilon = 0.02$, για τα πρώτα 100 βήματα}
%	\label{fig:untitled}
\end{figure}
\begin{figure}[htbp]
	\centerfloat
	\includegraphics{plots/algorithms/plot_c_bar.eps}
	\caption{Τιμές του $\overbar{x_k}$ της μεγίστης καθόδου με προβολή για $s_k = 20$, $\gamma_k = 0.3$, $\epsilon = 0.02$, για τα πρώτα 100 βήματα}
	\label{fig:c_xbar}
\end{figure}
\subsection{Ερώτημα (δ)}


Χρησιμοποιούμε τα ορίσματα $s_k = 0.1$, $\gamma_k = 0.01$, $\epsilon = 0.01$ και αρχικό σημείο το $(11,3)$.
Οι σχέσεις \ref{eq:x1-final} και \ref{eq:x2-final} γίνονται:
\begin{align}
	\label{eq:x1-finald}
	x_{1_{k+1}} =		
		\begin{cases}
			0.99 \cdot x_{1_{k}} - 0.2, &$ αν $ x_{1_{k}} \leq -\frac{200}{9} \\
			0.999 \cdot x_{1_{k}}, &$ αν $ -\frac{200}{9} < x_{1_{k}} < \frac{100}{9}\\
			0.99 \cdot x_{1_{k}} + 0.1, &$ αν $ x_{1_{k}} \geq \frac{100}{9}
		\end{cases}
\end{align}
και
\begin{align}
	\label{eq:x2-finald}
	x_{2_{k+1}} = 	
		\begin{cases}
			0.99 \cdot x_{2_{k}} - 0.12, &$ αν $ x_{2_{k}} \leq -\frac{40}{3} \\
			0.999 \cdot x_{2_{k}}, &$ αν $ -\frac{40}{3} < x_{2_{k}} < \frac{50}{3} \\
			0.99 \cdot x_{2_{k}} + 0.15, &$ αν $ x_{2_{k}} \geq \frac{50}{3}
		\end{cases}
\end{align}

Παρατηρούμε ότι άν το $0.9 \cdot x_k \in X$ τότε $x_{k+1} = 0.999 \cdot x_k$. 
Άρα αν $0.9 \cdot x_k \in X$ τότε $0.9 \cdot x_{k+1} \in X$. 
Έτσι, αν για κάποιο $k$ έχουμε $x_k \in X$ τότε $\forall k' > k$ θα ισχύει $x_{k'} \in X$.
Τα $s$ και $\gamma$ ικανοποιούν την σχέση \ref{eq:s-gamma-requirements}

Μάλιστα, για το αρχικό σημείο $x_1 = (11,3)$ ισχύει $\overbar{x_1} = \Pr\{0.9 \cdot x_1\}$. Όμως $0.9 \cdot x_1 = (9.9, 2.7) \in X$ και άρα $\overbar{x_1} = 0.999 \cdot x_k = (10.989, 2.997)$.
Δηλαδή, θα ισχύει $\overbar{x_k} = 0.999 \cdot x_k \forall k$.
Έτσι, εκ των προτέρων, μπορούμε να καταλάβουμε ότι ο αλγόριθμος θα συγκλίνει και να υπολογίσουμε τον αριθμό των βημάτων.

Έχουμε
\begin{gather}
	x_{k+1} = (1 - \gamma) \cdot x_k + \gamma \cdot (1-s) \cdot x_k =\nonumber\\
	x_k \cdot (1 - s \cdot \gamma)
\end{gather}

\begin{varwidth}{\textwidth}
Οπότε, για αρχικό σημείο $x_1$ η ακολουθία $\{x_k\}$ έχει ως εξής:
\begin{itemize}
	\item $x_2 = (1 - s \cdot \gamma) \cdot x_1$
	\item $x_3 = (1 - s \cdot \gamma) \cdot x_2 = (1 - s \cdot \gamma)^2 \cdot x_1$
	\item $x_4 = (1 - s \cdot \gamma) \cdot x_3 = (1 - s \cdot \gamma)^3 \cdot x_1$
	\item \ldots
	\item $x_k = (1 - s \cdot \gamma)^{k-1} \cdot x_1$
\end{itemize}
\end{varwidth}
\hfill \break
\hfill \break

Ισχύει
\begin{gather*}
	\norm{x_{k+1} - \overbar{x_k}} = \\
	\norm{(1-\gamma) \cdot x_k + \gamma \cdot \overbar{x_k} - \overbar{x_k}} = \\
	\norm{ (1-\gamma) \cdot x_k + (\gamma - 1) \cdot \overbar{x_k} } = \\
	\norm{ (1-\gamma) - (1 - \gamma) \cdot (1 -s) \cdot x_k} = \\
	\norm{ s \cdot (1-\gamma) \cdot x_k } = \\
	\abs{s \cdot (1-\gamma) \cdot (1 - s \cdot \gamma)^{k-1}} \cdot \norm{x_1} 
\end{gather*}

Από την συνθήκη τερματισμού \ref{eq:termination-condition} και αντικαθιστώντας τις τιμές για $\epsilon$, $\gamma$, $s$:
\begin{gather*}
	\abs{s \cdot (1-\gamma) \cdot (1 - s \cdot \gamma)^{k-1}} \cdot \norm{x_1} = \epsilon \implies\\
	(1 - s \cdot \gamma) ^ {k - 1} = \frac{\epsilon}{s \cdot \abs{1-\gamma} \cdot \norm{x_1}} \implies\\
	k = \frac{\ln{\frac{\epsilon}{s \cdot \abs{1-\gamma} \cdot \norm{x_1}}}}{\ln{\abs{1 - s \cdot \gamma}}} + 1 \implies\\
	k =  4725.938 \implies\\
	k = 4726
\end{gather*}

Αν εκτελέσουμε τον αλγόριθμο επιβεβαιώνονται αυτά τα αποτελέσματα αφού ο αυτός τερματίζει για $k = 4726$. Αυτή η ταχύτητα σύγκλισης είναι σχετικά αργή και για να την βελτιώσουμε θα πρέπει να αυξήσουμε τις τιμές του $\gamma$ και $s$ οι οποίες όμως πρέπει να τηρούν την σχέση \ref{eq:s-gamma-requirements}.

\begin{figure}[htbp]
	\centerfloat
	\includegraphics{plots/algorithms/plot_d.eps}
	\caption{Σύγκλιση της μεγίστης καθόδου με προβολή για $s_k = 0.1$, $\gamma_k = 0.01$, $\epsilon = 0.01$}
\end{figure}


\end{document}
