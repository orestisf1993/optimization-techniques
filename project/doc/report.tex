%&preamble
% used for figures in multicol enviroment
\usepackage{multicol}
\newenvironment{Figure}
  {\par\medskip\noindent\minipage{\linewidth}}
  {\endminipage\par\medskip}

% https://www.ctan.org/pkg/tabu
\usepackage{longtable}
%% end of static part. Save at preamble.tex and use command:
% xelatex -ini -shell-escape -job-name="preamble" "&xelatex preamble.tex\dump"
%% dynamic part or stuff that can't be in the precompiled file
%\usetikzlibrary{external}
%\tikzset{external/system call={xelatex -fmt=preamble.fmt \tikzexternalcheckshellescape -halt-on-error -interaction=batchmode -jobname "\image" "\texsource"}}
%\tikzexternalize
%delete external pdf if old
%\newcommand{\deloldext}[2]{%  
%   \immediate\write18{./deloldext.pl '#1' '#2'}
%}
%%include tikz file and call \deloldext
%\newcommand{\includetikz}[1]{%
%    \tikzsetnextfilename{#1}%
%    \deloldext{#1.tikz}{#1.pdf}%
%    \input{#1.tikz}%
%}

% Only for xelatex and lualatex. It provides an automatic and unified interface to feature-rich AAT and OpenType fonts.
% https://ctan.org/pkg/fontspec
\usepackage{fontspec}
\setmainfont{DejaVu Serif}
% Change the title of contents list and figure list
\renewcommand{\contentsname}{Περιεχόμενα}
\renewcommand{\listfigurename}{Λίστα Σχημάτων}
% Change the name of each figure
\renewcommand{\figurename}{Σχήμα}
\renewcommand{\lstlistingname}{Καταχώρηση}% Listing -> Algorithm
\renewcommand{\lstlistlistingname}{List of \lstlistingname s}


% defines a variable
%\def \variableName {Something that's better to use as a variable}

% \\ or \newline == newline, no paragraph
% \\* == no page break

\title{Τεχνικές Βελτιστοποίησης\\Project}
\author{Ορέστης Φλώρος-Μαλιβίτσης\\
  Τομέας Ηλεκτρονικής,\\
  Τμήμα Ηλ. Μηχανικών / Μηχανικών ΗΥ,\\
  Αριστοτέλειο Πανεπιστήμιο Θεσσαλονίκης}
\date{05/07/2015}
%\includeonly{measurements} % delete for release version

\begin{document}
\maketitle
\tableofcontents
%\listoffigures
\newpage

% Δομή του Project
\chapter*{Δομή του Project} \label{project-structure}

\begin{description}
	\item[Floros\_orestis\_Project\_7796.pdf] Αυτή η αναφορά.
	\item[thema.pdf] Η εκφώνηση του project.
	\item[error\_sum.m] Άθροισμα τετραγώνων των error.
	\item[f.m] Η συνάρτηση που δίνεται.
	\item[gaussianSum.m] Άθροισμα γκαουσιανών συναρτήσεων.
	\item[run\_all.m] Εκτέλεση των γενετικών.
\end{description}

% Εισαγωγή
\chapter{Εισαγωγή} \label{intro}

Δίνονται οι συναρτήσεις για ελαχιστοποίηση:
\begin{equation}
	\label{eq:f-definition}
	f:\mathbb{R}^{2} \mapsto \mathbb{R}, \qquad f(x_1, x_2) = x_1 \cdot x_2 + 2 \cdot (x_1 - x_2)^2
\end{equation}
με περιορισμούς
\begin{align}
	3 \leq x_1& \leq 30\\
	-25 \leq x_2& \leq -50
\end{align}
και
\begin{equation}
	\label{eq:g-definition}
	g:\mathbb{R}^{2} \mapsto \mathbb{R}, \qquad g(x_1, x_2) = (x_1 - x_2)^2
\end{equation}
με περιορισμούς
\begin{align}
	x_1& \leq -1\\
	x_2& \leq -1
\end{align}
Υπολογίζονται οι κλίσεις τους:

\begin{equation}
\nabla f(x_1, x_2) = 
\begin{bmatrix}
	4 \cdot x_1 - 3 \cdot x_2\\
	-3 \cdot x_1 + 4 \cdot x_2
\end{bmatrix}
\end{equation}
και
\begin{equation}
\nabla g(x_1, x_2) = 
\begin{bmatrix}
	2 \cdot (x_1 - x_2)\\
	-2 \cdot (x_1 - x_2)
\end{bmatrix}
\end{equation}
Οι Εσσιανοί πίνακες υπολογίζονται ως:
\begin{equation}
\nabla^2 f(x_1, x_2) = 
\begin{bmatrix}
	4 & -3\\
	-3 & 4
\end{bmatrix}
\end{equation} με ιδιοτιμές $\lambda_1 = 7$ και $\lambda_2 = 1$.
\begin{equation}
\nabla^2 f(x_1, x_2) = 
\begin{bmatrix}
	2 & -2\\
	-2 & 2
\end{bmatrix}
\end{equation} με ιδιοτιμές $\lambda_1 = 4$ και $\lambda_2 = 0$.\\
Παρατηρούμε ότι λόγω των Εσσιανών τους πινάκων οι $f$ και $g$ είναι κυρτές συναρτήσεις και τα αντίστοιχα προβλήματα είναι κυρτά και υπερ-αποτελούμενα.\\
Παρουσιάζονται σχετικές γραφικές παραστάσεις.

\setlength{\LTleft}{-20cm plus -1fill}
\setlength{\LTright}{\LTleft}

\section{Συνάρτηση f}
\begin{longtable}{cc}
	\centering
	\includegraphics[width=100mm]{plots/functions/f_mesh_plot.eps}&
	\includegraphics[width=100mm]{plots/functions/f_mesh_plot_limited.eps}\\
	\includegraphics[width=100mm]{plots/functions/f_contourf_plot.eps} &
	\includegraphics[width=100mm]{plots/functions/f_contourf_plot_limited.eps}\\
	\includegraphics[width=100mm]{plots/functions/gradf_mesh_plot.eps} &
	\includegraphics[width=100mm]{plots/functions/gradf_mesh_plot_limited.eps}\\
\end{longtable}

\newpage
\section{Συνάρτηση g}
\setlength{\LTleft}{-20cm plus -1fill}
\setlength{\LTright}{\LTleft}
\begin{longtable}{cc}
	\centerfloat
	\includegraphics[width=100mm]{plots/functions/g_mesh_plot.eps} &
	\includegraphics[width=100mm]{plots/functions/g_mesh_plot_limited.eps}\\
	\includegraphics[width=100mm]{plots/functions/g_contourf_plot.eps} &
	\includegraphics[width=100mm]{plots/functions/g_contourf_plot_limited.eps}\\
	\includegraphics[width=100mm]{plots/functions/gradg_mesh_plot.eps} &
	\includegraphics[width=100mm]{plots/functions/gradg_mesh_plot_limited.eps}\\
\end{longtable}
\chapter{Υλοποίηση στο Matlab} \label{ch:matlab}

Για λόγους γρηγορότερης εκτέλεσης τα αρχεία είναι vectorized,
δηλαδή μπορούν αν δέχονται σαν ορίσματα διανύσματα αντί για βαθμωτά μεγέθη.

\section{gaussianSum.m}

\lstinputlisting[language=Matlab, caption=gaussianSum.m]{../gaussianSum.m}

Ο κώδικας για την υλοποίηση του αθροίσματος των γκαουσιανών συναρτήσεων βρίσκεται στο αρχείο gaussianSum.m. Επιστρέφεται η τιμή 
\begin{equation}
	z = \sum_{i = 1}^{nGaussian} a_i \cdot e^{-[\frac{(u_1 - c_{1_i})^2}{2\sigma_{1_i}^2} + \frac{(u_2 - c_{2_i})^2}{2\sigma_{2_i}^2}]}
\end{equation}

Το όρισμα $x$ περιέχει το διάνυσμα με τα μεγέθη που χρησιμοποιούνται για τον προσδιορισμό των συναρτήσεων 
και είναι αυτό που θα προσπαθήσουμε να βελτιστοποιήσουμε μέσω της ga().

Αν ορίζουμε ως $nGaussian$ τον αριθμό των γκαουσιανών συναρτήσεων που χρησιμοποιούμε, 
τότε τα πρώτα $nGaussian$ στοιχεία του $x$ είναι οι συντελεστές $a_i$, 
τα επόμενα $2 \cdot nGaussian$ στοιχεία είναι οι συντελεστές $\sigma_{1_i}$ και $\sigma_{2_i}$ 
και τα τελευταία $2 \cdot nGaussian$ στοιχεία είναι οι συντελεστές $c_{1_i}$ και $c_{2_i}$.

Λόγω του vectorization της συνάρτησης ως προς $x$ αλλά και ως προς $u_1$, $u_2$ χρησιμοποιείται η συνάρτηση bsxfun() του Matlab ώστε να γίνονται πράξεις στοιχείο προς στοιχείο για ασύμμετρους πίνακες.

\section{error\_sum.m}

\lstinputlisting[language=Matlab, caption=error\_sum.m]{../error_sum.m}

Σχετικά απλή συνάρτηση που αθροίζει τα τετράγωνα των διαφορών $f(u_1, u_2)$ και $gaussianSum()$.

\section{run\_all.m}

\lstinputlisting[language=Matlab, caption=run\_all.m]{../run_all.m}

Στη φάση του initialization ορίζουμε τα διανύσματα $u_1$, $u_2$ και θέτουμε τα όρια του διανύσματος $x$. 
Συγκεκριμένα, τα $\sigma_{i}$ επιλέγονται θετικά και μικρότερα του $2\sqrt{2}$. 
Η επιλογή αυτή έγινε καθώς τα $\sigma$ σχετίζεται ουσιαστικά με το πλάτος της γκαουσιανής "καμπάνας" ως προς $u_1$ και $u_2$ και δεν θα είχε νόημα αν αυτή ήταν μεγαλύτερη από το πεδίο ορισμού $[-2, 2]$. 
Θέλουμε τουλάχιστον η μισή "καμπάνα" να είναι μέσα στο πεδίο ορισμού έχουμε $\sigma > 0$ και $\frac{\sigma^2}{2} \leq 2 - (-2) \implies 0 < \sigma \leq 2\sqrt{2}$. 
Επίσης, επιλέγουμε τα $c_i$ που καθορίζουν την θέση των κορυφών τέτοια ώστε να είναι μέσα στο πεδίο ορισμού, δηλαδή $-2 \leq c \leq 2$. Δοκιμάστηκαν και μεγαλύτερα όρια ($-6 \leq c \leq 6$) αλλά είχαν χειρότερα αποτελέσματα.
Τα $a_i$ δεν έχουν όρια.

Στη συνέχεια, καλούμε την ga() δοκιμάζοντας αριθμό γκαουσιανών από 1 έως 15.
Στην κάθε κλίση της ga() ελαχιστοποιούμε ως προς ένα διάνυσμα $x$ $5 \cdot nGaussian$ στοιχείων. 
Τα $u_1$ και $u_2$ μένουν σταθερά σύμφωνα με τον αρχικό υπολογισμό τους.

Τελικά, επιλέγουμε έναν μια από τις τιμές $nGaussian$ και προσπαθούμε να βελτιώσουμε την προσέγγιση περαιτέρω ορίζοντας επαναληπτικά ως αρχικό πληθυσμό της ga() το προηγούμενο αποτέλεσμα $x$.
\chapter{Μετρήσεις και αποτελέσματα} \label{ch:meashurements}

Λόγω της τυχαιότητας των γενετικών αλγορίθμων τα αποτελέσματα που βρίσκουμε δεν είναι σταθερά και ο αριθμός των γκαουσιανών για τον οποίο έχουμε μικρότερο error αλλάζει κάθε φορά που εκτελούμε τον κώδικα.
\newline

\begin{tabular}{ c | c }
   nGaussian & error \\
   \hline
	1 & 128.9957 \\ 
	2 & 61.6691 \\ 
	3 & 50.8957 \\ 
	4 & 35.9179 \\ 
	5 & 42.1853 \\ 
	6 & 48.8752 \\ 
	7 & 23.265 \\ 
	8 & 11.9985 \\ 
	9 & 20.0387 \\ 
	10 & 14.7827 \\ 
	11 & 61.483 \\ 
	12 & 20.3933 \\ 
	13 & 13.0228 \\ 
	14 & 6.3402 \\ 
	15 & 6.3388 \\ 
\end{tabular}
\captionof{table}{Αποτελέσματα για το error σε σχέση με τον αριθμό των γκαουσιανών}

\newpage
Παρουσιάζονται οι γραφικές παραστάσεις των προσεγγίσεων.
\begin{multicols}{2}

\begin{Figure}
	\includegraphics[width=5.5cm]{adapted_1}
	\label{fig:adapted-1}
\end{Figure}
\begin{Figure}
	\includegraphics[width=5.5cm]{adapted_2}
	\label{fig:adapted-2}
\end{Figure}
\begin{Figure}
	\includegraphics[width=5.5cm]{adapted_3}
	\label{fig:adapted-3}
\end{Figure}
\begin{Figure}
	\includegraphics[width=5.5cm]{adapted_4}
	\label{fig:adapted-4}
\end{Figure}
\begin{Figure}
	\includegraphics[width=5.5cm]{adapted_5}
	\label{fig:adapted-5}
\end{Figure}
\begin{Figure}
	\includegraphics[width=5.5cm]{adapted_6}
	\label{fig:adapted-6}
\end{Figure}
\begin{Figure}
	\includegraphics[width=5.5cm]{adapted_7}
	\label{fig:adapted-7}
\end{Figure}
\begin{Figure}
	\includegraphics[width=5.5cm]{adapted_8}
	\label{fig:adapted-8}
\end{Figure}
\begin{Figure}
	\includegraphics[width=5.5cm]{adapted_9}
	\label{fig:adapted-9}
\end{Figure}
\begin{Figure}
	\includegraphics[width=5.5cm]{adapted_10}
	\label{fig:adapted-10}
\end{Figure}
\begin{Figure}
	\includegraphics[width=5.5cm]{adapted_11}
	\label{fig:adapted-11}
\end{Figure}
\begin{Figure}
	\includegraphics[width=5.5cm]{adapted_12}
	\label{fig:adapted-12}
\end{Figure}
\begin{Figure}
	\includegraphics[width=5.5cm]{adapted_13}
	\label{fig:adapted-13}
\end{Figure}
\begin{Figure}
	\includegraphics[width=5.5cm]{adapted_14}
	\label{fig:adapted-14}
\end{Figure}

\end{multicols}

\begin{figure}[bh]
	\centerfloat
	\includegraphics[width=8cm]{adapted_15}
	\label{fig:adapted-15}
\end{figure}

\newpage
Παρουσιάζονται οι γραφικές παραστάσεις των error.
\begin{multicols}{2}

\begin{Figure}
	\includegraphics[width=5.5cm]{adapted_error_1}
	\label{fig:adapted-error-1}
\end{Figure}
\begin{Figure}
	\includegraphics[width=5.5cm]{adapted_error_2}
	\label{fig:adapted-error-2}
\end{Figure}
\begin{Figure}
	\includegraphics[width=5.5cm]{adapted_error_3}
	\label{fig:adapted-error-3}
\end{Figure}
\begin{Figure}
	\includegraphics[width=5.5cm]{adapted_error_4}
	\label{fig:adapted-error-4}
\end{Figure}
\begin{Figure}
	\includegraphics[width=5.5cm]{adapted_error_5}
	\label{fig:adapted-error-5}
\end{Figure}
\begin{Figure}
	\includegraphics[width=5.5cm]{adapted_error_6}
	\label{fig:adapted-error-6}
\end{Figure}
\begin{Figure}
	\includegraphics[width=5.5cm]{adapted_error_7}
	\label{fig:adapted-error-7}
\end{Figure}
\begin{Figure}
	\includegraphics[width=5.5cm]{adapted_error_8}
	\label{fig:adapted-error-8}
\end{Figure}
\begin{Figure}
	\includegraphics[width=5.5cm]{adapted_error_9}
	\label{fig:adapted-error-9}
\end{Figure}
\begin{Figure}
	\includegraphics[width=5.5cm]{adapted_error_10}
	\label{fig:adapted-error-10}
\end{Figure}
\begin{Figure}
	\includegraphics[width=5.5cm]{adapted_error_11}
	\label{fig:adapted-error-11}
\end{Figure}
\begin{Figure}
	\includegraphics[width=5.5cm]{adapted_error_12}
	\label{fig:adapted-error-12}
\end{Figure}
\begin{Figure}
	\includegraphics[width=5.5cm]{adapted_error_13}
	\label{fig:adapted-error-13}
\end{Figure}
\begin{Figure}
	\includegraphics[width=5.5cm]{adapted_error_14}
	\label{fig:adapted-error-14}
\end{Figure}

\end{multicols}

\begin{figure}[bh]
	\centerfloat
	\includegraphics[width=8cm]{adapted_error_15}
	\label{fig:adapted-error-15}
\end{figure}

\hfill \break
Για να μην αυξήσουμε πολύ την πολυπλοκότητα της έκφρασης της $f$ διαλέγουμε έναν σχετικά μικρό αριθμό γκαουσιανών για να μελετήσουμε παραπάνω. 
Για $nGaussian = 8$ επαναλαμβάνουμε την εύρεση ελαχίστου μέχρι που ρίχνουμε το error από $11.9985$ σε $10.6045$. 
Τα νέα αποτελέσματα φαίνονται στα γραφήματα \ref{fig:adapted-opt-8} και \ref{fig:adapted-opt-error-8}. 
Ωστόσο, όπως φαίνεται στα γραφήματα \ref{fig:adapted-actual-8} και \ref{fig:adapted-error-actual-8} παρατηρείται μια κορύφωση του σφάλματος, 
που δεν φαίνεται στα προηγούμενα γραφήματα, 
όταν χρησιμοποιείται διαφορετικό, πυκνότερο σύνολο σημείων για τον σχεδιασμό του γραφήματος.

\begin{multicols}{2}

\begin{Figure}
	\centerfloat
	\includegraphics[width=5.5cm]{adapted_opt_8}
	\captionof{figure}{Βελτιωμένη προσέγγιση 8 γκαουσιανών}
	\label{fig:adapted-opt-8}
\end{Figure}

\begin{Figure}
	\centerfloat
	\includegraphics[width=5.5cm]{adapted_opt_error_8}
	\captionof{figure}{error βελτιωμένης προσέγγιση 8 γκαουσιανών}
	\label{fig:adapted-opt-error-8}
\end{Figure}

\begin{Figure}
	\centerfloat
	\includegraphics[width=5.5cm]{adapted_actual_8}
	\captionof{figure}{Βελτιωμένη προσέγγιση 8 γκαουσιανών με διαφορετικά σημεία}
	\label{fig:adapted-actual-8}
\end{Figure}

\begin{Figure}
	\centerfloat
	\includegraphics[width=5.5cm]{adapted_actual_error_8}
	\captionof{figure}{error βελτιωμένης προσέγγιση 8 γκαουσιανών με διαφορετικά σημεία}
	\label{fig:adapted-error-actual-8}
\end{Figure}

\end{multicols}


\end{document}
