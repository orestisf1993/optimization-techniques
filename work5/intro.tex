% Εισαγωγή
\chapter{Εισαγωγή} \label{intro}

Δίνονται οι συναρτήσεις για ελαχιστοποίηση:
\begin{equation}
	\label{eq:f-definition}
	f:\mathbb{R}^{2} \mapsto \mathbb{R}, \qquad f(x_1, x_2) = x_1 \cdot x_2 + 2 \cdot (x_1 - x_2)^2
\end{equation}
με περιορισμούς
\begin{align}
	3 \leq x_1& \leq 30\\
	-25 \leq x_2& \leq -50
\end{align}
και
\begin{equation}
	\label{eq:g-definition}
	g:\mathbb{R}^{2} \mapsto \mathbb{R}, \qquad g(x_1, x_2) = (x_1 - x_2)^2
\end{equation}
με περιορισμούς
\begin{align}
	x_1& \leq -1\\
	x_2& \leq -1
\end{align}
Υπολογίζονται οι κλίσεις τους:

\begin{equation}
\nabla f(x_1, x_2) = 
\begin{bmatrix}
	4 \cdot x_1 - 3 \cdot x_2\\
	-3 \cdot x_1 + 4 \cdot x_2
\end{bmatrix}
\end{equation}
και
\begin{equation}
\nabla g(x_1, x_2) = 
\begin{bmatrix}
	2 \cdot (x_1 - x_2)\\
	-2 \cdot (x_1 - x_2)
\end{bmatrix}
\end{equation}
Οι Εσσιανοί πίνακες υπολογίζονται ως:
\begin{equation}
\nabla^2 f(x_1, x_2) = 
\begin{bmatrix}
	4 & -3\\
	-3 & 4
\end{bmatrix}
\end{equation} με ιδιοτιμές $\lambda_1 = 7$ και $\lambda_2 = 1$.
\begin{equation}
\nabla^2 f(x_1, x_2) = 
\begin{bmatrix}
	2 & -2\\
	-2 & 2
\end{bmatrix}
\end{equation} με ιδιοτιμές $\lambda_1 = 4$ και $\lambda_2 = 0$.\\
Παρατηρούμε ότι λόγω των Εσσιανών τους πινάκων οι $f$ και $g$ είναι κυρτές συναρτήσεις και τα αντίστοιχα προβλήματα είναι κυρτά και υπερ-αποτελούμενα.\\
Παρουσιάζονται σχετικές γραφικές παραστάσεις.

\setlength{\LTleft}{-20cm plus -1fill}
\setlength{\LTright}{\LTleft}

\section{Συνάρτηση f}
\begin{longtable}{cc}
	\centering
	\includegraphics[width=100mm]{plots/functions/f_mesh_plot.eps}&
	\includegraphics[width=100mm]{plots/functions/f_mesh_plot_limited.eps}\\
	\includegraphics[width=100mm]{plots/functions/f_contourf_plot.eps} &
	\includegraphics[width=100mm]{plots/functions/f_contourf_plot_limited.eps}\\
	\includegraphics[width=100mm]{plots/functions/gradf_mesh_plot.eps} &
	\includegraphics[width=100mm]{plots/functions/gradf_mesh_plot_limited.eps}\\
\end{longtable}

\newpage
\section{Συνάρτηση g}
\setlength{\LTleft}{-20cm plus -1fill}
\setlength{\LTright}{\LTleft}
\begin{longtable}{cc}
	\centerfloat
	\includegraphics[width=100mm]{plots/functions/g_mesh_plot.eps} &
	\includegraphics[width=100mm]{plots/functions/g_mesh_plot_limited.eps}\\
	\includegraphics[width=100mm]{plots/functions/g_contourf_plot.eps} &
	\includegraphics[width=100mm]{plots/functions/g_contourf_plot_limited.eps}\\
	\includegraphics[width=100mm]{plots/functions/gradg_mesh_plot.eps} &
	\includegraphics[width=100mm]{plots/functions/gradg_mesh_plot_limited.eps}\\
\end{longtable}