% Εισαγωγή
\chapter{Εισαγωγή} \label{ch:intro}

Μελετάμε το σύστημα
\begin{equation}
	y = f(u_1, u_2)
\end{equation}
για του οποίου της μετρήσεις χρησιμοποιούμε την συνάρτηση
\begin{equation}
	f(u_1, u_2) = \sin{(u_1 + u_2)}\sin{(u_1^2)}
	\label{eq:f}
\end{equation}

\begin{figure}[htbp]
	\centerfloat	
	\includegraphics{f}
	\caption{Η συνάρτηση $f(u_1, u_2)$}
	\label {fig:f_mesh_plot}
\end{figure}

Στόχος μας είναι μέσω του γραμμικού συνδυασμού γκαουσιανών συναρτήσεων (Gaussian functions) της μορφής
\begin{equation}
	G(u_1, u_2) = a \cdot e^{-[\frac{(u_1 - c_1)^2}{2\sigma_{1}^2} + \frac{(u_2 - c_2)^2}{2\sigma_{2}^2}]}
\end{equation}
να προσεγγίσουμε τις μετρήσεις που παίρνουμε από την σχέση \ref{eq:f}.

\begin{figure}
	\centerfloat	
	\includegraphics[width=0.95\linewidth]{example_gaussian}
	\caption{Παράδειγμα γκαουσιανής συνάρτησης}
	\label {fig:example-gaussian}
\end{figure}

Για να το πετύχουμε αυτό χρησιμοποιούμε την συνάρτηση ga() του Matlab που βρίσκει το ελάχιστο μιας συνάρτησης χρησιμοποιώντας γενετικό αλγόριθμο. 
Θα ελαχιστοποιήσουμε το άθροισμα τετραγώνων της διαφοράς μεταξύ των μετρήσεων και των τιμών της γκαουσιανής προσέγγισης.
\begin{equation}
	E = \sum_{i = 1}^{nPoints}(f(u_{1_i}, u_{2_i}) - \sum_{j = 1}^{nGaussian}G_j(u_{1_i}, u_{2_i}))^2
\end{equation}