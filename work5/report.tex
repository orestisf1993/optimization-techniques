%&preamble
% https://www.ctan.org/pkg/tabu
\usepackage{longtable}
%% end of static part. Save at preamble.tex and use command:
% xelatex -ini -shell-escape -job-name="preamble" "&xelatex preamble.tex\dump"
%% dynamic part or stuff that can't be in the precompiled file
%\usetikzlibrary{external}
%\tikzset{external/system call={xelatex -fmt=preamble.fmt \tikzexternalcheckshellescape -halt-on-error -interaction=batchmode -jobname "\image" "\texsource"}}
%\tikzexternalize
%delete external pdf if old
%\newcommand{\deloldext}[2]{%  
%   \immediate\write18{./deloldext.pl '#1' '#2'}
%}
%%include tikz file and call \deloldext
%\newcommand{\includetikz}[1]{%
%    \tikzsetnextfilename{#1}%
%    \deloldext{#1.tikz}{#1.pdf}%
%    \input{#1.tikz}%
%}

% Only for xelatex and lualatex. It provides an automatic and unified interface to feature-rich AAT and OpenType fonts.
% https://ctan.org/pkg/fontspec
\usepackage{fontspec}
\setmainfont{DejaVu Serif}
% Change the title of contents list and figure list
\renewcommand{\contentsname}{Περιεχόμενα}
\renewcommand{\listfigurename}{Λίστα Σχημάτων}
% Change the name of each figure
\renewcommand{\figurename}{Σχήμα}
\renewcommand{\lstlistingname}{Καταχώρηση}% Listing -> Algorithm
\renewcommand{\lstlistlistingname}{List of \lstlistingname s}


% defines a variable
%\def \variableName {Something that's better to use as a variable}

% \\ or \newline == newline, no paragraph
% \\* == no page break

\title{Τεχνικές Βελτιστοποίησης\\5η Εργαστηριακή Άσκηση}
\author{Ορέστης Φλώρος-Μαλιβίτσης\\
  Τομέας Ηλεκτρονικής,\\
  Τμήμα Ηλ. Μηχανικών / Μηχανικών ΗΥ,\\
  Αριστοτέλειο Πανεπιστήμιο Θεσσαλονίκης}
\date{03/06/2015}

%\includeonly{part_a} % delete for release version

\begin{document}
\maketitle
\tableofcontents
%\listoffigures
\newpage

% Δομή του Project
\chapter*{Δομή του Project} \label{project-structure}

\begin{description}
	\item[Floros\_orestis\_Work5\_7796.pdf] Αυτή η αναφορά.
	\item[work5.pdf] Η εκφώνηση της εργαστηριακής άσκησης.
	\item[plots] Φάκελος που περιέχει τους κώδικες Matlab για την δημιουργία γραφικών παραστάσεων.
	\item[functions] Φάκελος με τις συναρτήσεις που πρέπει να ελαχιστοποιηθούν.
	\item[algorithms] Υλοποιήσεις των βοηθητικών συναρτήσεων και των μεθόδων φραγμού/ποινής.
\end{description}

% Εισαγωγή
\chapter{Εισαγωγή} \label{intro}

Δίνονται οι συναρτήσεις για ελαχιστοποίηση:
\begin{equation}
	\label{eq:f-definition}
	f:\mathbb{R}^{2} \mapsto \mathbb{R}, \qquad f(x_1, x_2) = x_1 \cdot x_2 + 2 \cdot (x_1 - x_2)^2
\end{equation}
με περιορισμούς
\begin{align}
	3 \leq x_1& \leq 30\\
	-25 \leq x_2& \leq -50
\end{align}
και
\begin{equation}
	\label{eq:g-definition}
	g:\mathbb{R}^{2} \mapsto \mathbb{R}, \qquad g(x_1, x_2) = (x_1 - x_2)^2
\end{equation}
με περιορισμούς
\begin{align}
	x_1& \leq -1\\
	x_2& \leq -1
\end{align}
Υπολογίζονται οι κλίσεις τους:

\begin{equation}
\nabla f(x_1, x_2) = 
\begin{bmatrix}
	4 \cdot x_1 - 3 \cdot x_2\\
	-3 \cdot x_1 + 4 \cdot x_2
\end{bmatrix}
\end{equation}
και
\begin{equation}
\nabla g(x_1, x_2) = 
\begin{bmatrix}
	2 \cdot (x_1 - x_2)\\
	-2 \cdot (x_1 - x_2)
\end{bmatrix}
\end{equation}
Οι Εσσιανοί πίνακες υπολογίζονται ως:
\begin{equation}
\nabla^2 f(x_1, x_2) = 
\begin{bmatrix}
	4 & -3\\
	-3 & 4
\end{bmatrix}
\end{equation} με ιδιοτιμές $\lambda_1 = 7$ και $\lambda_2 = 1$.
\begin{equation}
\nabla^2 f(x_1, x_2) = 
\begin{bmatrix}
	2 & -2\\
	-2 & 2
\end{bmatrix}
\end{equation} με ιδιοτιμές $\lambda_1 = 4$ και $\lambda_2 = 0$.\\
Παρατηρούμε ότι λόγω των Εσσιανών τους πινάκων οι $f$ και $g$ είναι κυρτές συναρτήσεις και τα αντίστοιχα προβλήματα είναι κυρτά και υπερ-αποτελούμενα.\\
Παρουσιάζονται σχετικές γραφικές παραστάσεις.

\setlength{\LTleft}{-20cm plus -1fill}
\setlength{\LTright}{\LTleft}

\section{Συνάρτηση f}
\begin{longtable}{cc}
	\centering
	\includegraphics[width=100mm]{plots/functions/f_mesh_plot.eps}&
	\includegraphics[width=100mm]{plots/functions/f_mesh_plot_limited.eps}\\
	\includegraphics[width=100mm]{plots/functions/f_contourf_plot.eps} &
	\includegraphics[width=100mm]{plots/functions/f_contourf_plot_limited.eps}\\
	\includegraphics[width=100mm]{plots/functions/gradf_mesh_plot.eps} &
	\includegraphics[width=100mm]{plots/functions/gradf_mesh_plot_limited.eps}\\
\end{longtable}

\newpage
\section{Συνάρτηση g}
\setlength{\LTleft}{-20cm plus -1fill}
\setlength{\LTright}{\LTleft}
\begin{longtable}{cc}
	\centerfloat
	\includegraphics[width=100mm]{plots/functions/g_mesh_plot.eps} &
	\includegraphics[width=100mm]{plots/functions/g_mesh_plot_limited.eps}\\
	\includegraphics[width=100mm]{plots/functions/g_contourf_plot.eps} &
	\includegraphics[width=100mm]{plots/functions/g_contourf_plot_limited.eps}\\
	\includegraphics[width=100mm]{plots/functions/gradg_mesh_plot.eps} &
	\includegraphics[width=100mm]{plots/functions/gradg_mesh_plot_limited.eps}\\
\end{longtable}
% part_a
\chapter{Θεωρητική λύση με χρήση του θεωρήματος Karush–Kuhn–Tucker} \label{part_a}

\section{Συνάρτηση f}

Για το πρόβλημα
\begin{equation}
	\label{eq:f-problem}
	\min_{
		\substack{3 \leq x_1 \leq 30 \\\\ -25 \leq x_2 \leq -5}
		}
	f(x_1, x_2)
\end{equation}
Οι περιορισμοί γράφονται ως εξής:
\begin{align*}
	g_{1_f} = 3 - x_1 &\leq 0\\
	g_{2_f} = x_1 - 30 &\leq 0\\
	g_{3_f} = -x_2 -25 &\leq 0\\
	g_{4_f} = x_2 + 5 &\leq 0
\end{align*}
και οι κλίσεις τους υπολογίζονται ως:
\begin{align*}
	\nabla g_{1_f} =& 
		\begin{bmatrix}
			-1& 0
		\end{bmatrix}^T\\
	\nabla g_{2_f} =& 
		\begin{bmatrix}
			1& 0
		\end{bmatrix}^T\\
	\nabla g_{3_f} =& 
		\begin{bmatrix}
			0& -1
		\end{bmatrix}^T\\
	\nabla g_{4_f} =& 
		\begin{bmatrix}
			0& 1
		\end{bmatrix}^T\\
\end{align*}
Τα κριτήρια του Karush-Kuhn-Tucker γράφονται ως εξής:
\begin{enumerate}[i)]
	\item $\lambda_1 \geq 0$, $\lambda_2 \geq 0$, $\lambda_3 \geq 0$, $\lambda_4 \geq 0$
	\item \begin{align*}
			\lambda_1 \cdot (x_1 - 3) = 0\\
			\lambda_2 \cdot (x_1 - 30) = 0\\
			\lambda_3 \cdot (x_2 + 25) = 0\\
			\lambda_4 \cdot (x_2 + 5) = 0
		\end{align*}
	\item \begin{equation*}
		\begin{bmatrix}
			4 \cdot x_1 - 3 \cdot x_2\\
			-3 \cdot x_1 + 4 \cdot x_2
		\end{bmatrix} + 
		\begin{bmatrix}
			-\lambda_1\\
			0
		\end{bmatrix} +
		\begin{bmatrix}
			\lambda_2\\
			0
		\end{bmatrix} +
		\begin{bmatrix}
			0\\
			-\lambda_3
		\end{bmatrix} +		
		\begin{bmatrix}
			0\\
			\lambda_4
		\end{bmatrix} = 
		\begin{bmatrix}
			0\\
			0
		\end{bmatrix}
	\end{equation*}
\end{enumerate}
Λύνοντας τα παραπάνω συστήματα παίρνουμε τις λύσεις
\begin{equation}
	(x_1^*, x_2^*) = (3, -5)
\end{equation}
με διάνυσμα πολλαπλασιαστών Lagrange:
\begin{equation*}
	(\lambda_1^*, \lambda_2^*, \lambda_3^*, \lambda_4^*) = (27, 0, 0, 29)
\end{equation*}

\newpage
\section{Συνάρτηση g}

Για το πρόβλημα
\begin{equation}
	\label{eq:g-problem}
	\min_{
		\substack{x_1 \leq -1 \\ x_2 \leq -1}
		}
	g(x_1, x_2)
\end{equation}
Οι περιορισμοί γράφονται ως εξής:
\begin{align*}
	g_{1_f} = x_1 + 1 &\leq 0\\
	g_{2_f} = x_2 + 1 &\leq 0
\end{align*}
και οι κλίσεις τους υπολογίζονται ως:
\begin{align*}
	\nabla g_{1_f} =& 
		\begin{bmatrix}
			1& 0
		\end{bmatrix}^T\\
	\nabla g_{2_f} =& 
		\begin{bmatrix}
			0& 1
		\end{bmatrix}^T
\end{align*}
Τα κριτήρια του Karush-Kuhn-Tucker γράφονται ως εξής:
\begin{enumerate}[i)]
	\item $\lambda_1 \geq 0$, $\lambda_2 \geq 0$
	\item \begin{align*}
			\lambda_1 \cdot (x_1 + 1) = 0\\
			\lambda_2 \cdot (x_2 + 1) = 0
		\end{align*}
	\item \begin{equation*}
		\begin{bmatrix}
			2 \cdot x_1 - 2 \cdot x_2\\
			-2 \cdot x_1 + 2 \cdot x_2
		\end{bmatrix} + 
		\begin{bmatrix}
			\lambda_1\\
			0
		\end{bmatrix} +
		\begin{bmatrix}
			0\\
			\lambda_2
		\end{bmatrix} = 
		\begin{bmatrix}
			0\\
			0
		\end{bmatrix}
	\end{equation*}
\end{enumerate}

Η λύση αυτού του συστήματος προκύπτει
\begin{equation}
	(x_1^*, x_2^*) = (t, t) \qquad t \leq -1
\end{equation}
Δηλαδή, το πρόβλημα \ref{eq:g-problem} δεν έχει μοναδική λύση καθώς η $g(x_1, x_2)$ δεν είναι γνησίως κυρτή.

\end{document}
